\Chapter{Folyamatok az iparban}
Minden nagyvállalatban egyszerre akár több üzleti, logisztikai folyamat zajlik. Ezeket a folyamatokat többféleképpen nyomon lehet követni. \\
\Section{Bpm bemutatása}
% TODO: Leírni, hogy a megközelítési módok miben különbözhetnek.
A legtöbb nagy cég a BPM (Business Process Management) platformot használja a feladatok nyomonkövetésére, menedzselésére.
A rendszerben megtalálható Process Designer program segítségével minden folyamathoz tartozik egy folyamatábra.
\\A folyamatábrán az elvégzett feladatokat zölddel, az aktuális lépést pirossal, a következő lépéseket pedig szürkével jelöli.
A rendszer listázza az összes meglévő folyamatot, amelyet el lehet indítani egy cégnél.
A folyamatábrán az egyes lépésekre kattintva további részletes információkat tekinthetünk meg, mint a view pagen használt mezőket,
a mögötte lévő adatbázis mezőket, a felhasznált ajax service-eket. Az Ajax Service implementációit lehet megtekinteni.
Az egyes elágazásokban lévő döntési változókat lehet megtekinteni stb. (Elég részletes program)
\Section{Bpm rendszer korlátja, hátrányai}
Ez a rendszer fizetős, nem érhetőek el ezek a szolgáltatások Lokális gépről.
A rendszer mivel elég széles körü, így eléggé rugalmatlan is, mivel ha hirtelen egy új üzleti folyamatra lenne szüksége a cégnek, akkor annak az
elkészítése akár több hetet is igénybe vehet.

Hogy példát említsek üzleti folyamatokra:
Egy autógyárban lehet akár egy gyártási folyamat, akár egy karbantartási folyamat, vagy valamilyen igénylési folyamat, amelyek több lépésekben folynak le.
Több jogosultsági szintek vannak egy ilyen folyamatnál, amelyeket megkell különböztetni. Mikor, melyik lépésnél ki a jóváhagyó stb.
Ezeket a folyamatokat valahol nyilván kell tartani, kezelni kell őket.

Folyamatok előfordulhatnak ipait vállalatoknál (Bosch, Joyson, BorsodChem),\\ és  egyéb logisztikával kapcsolatos cégeknél, különböző kereskedelmi szervezeteknél.
A legtöbb nagyipari vállalat az IBM-BPM platformot használja.
A folyamatok menedzseléséhez, követéséhez hozzátartozik az SAP, amely egy vállalatirányítási rendszer, sokszor indítanak el SAP-ból egy folyamatot.
Ahhoz, hogy egy folyamatot elkészítsünk, amely sap-ból indul, szükséges a Process Designerben ezt Ajax Service formájában definiálni, és Postman segítségével teszteleni,
amely a legtöbb esetben egy folyamat azonosítót és egy lépés azonosítót ad vissza. Ezeket megkeresve tudjuk az adatbázisba megtekinteni , felhasználni, és a formon kiolvasni az adatokat.


-Egy karbantartási folyamat esetén , egy nagyobb cégnél 5 lépésben folyik le:
\\-Az igénylő kitölti a formot, ahol megadja a gépjármű adatait, a probléma pontos leírását. Továbbítja a folyamatot.
\\-Az első jóváhagyó (SupervisorApprove) ez a formot megkapja, ír hozzá egy megjegyzést, majd eztkövetően a folyamat továbbmegy a coordinátorhoz,
amely meghatározza a probléma leírása alapján,a költségeket, a javítás várható idő intervallumát, a helyet stb.
\\-Amennyiben költség szintje meghaladja az adott értéket, akkor a folyamat folytatásához a Director jóváhagyása kell. Ő dönti el hogy vissza reject-eli, vagy továbbítja a folyamatot.
\\-A feladat visszakerül az igénylőhöz