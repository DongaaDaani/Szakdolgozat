\Chapter{A felhasználói felület kialakítása}

Ebben a fejezetben bemutatásra kerül a folyamatmenedzsment rendszer
felhasználói felülete. %, illetve az adatbázis modellje.
A felhasználói felületet
képernyőfotókkal kiegészítve, egy kisebb változatú, de hasonló célú projekttel
mutatom be, illetve mindegyikhez adok egy-egy rövid leírást az adott felhasználói felület funkcióiról.
% A programunkhoz használt adatbázis leírása és bemutatása után  részletesen bemutatjuk a programunk adatmodelljét, ER és relációs modell segítségével.

\Section{Felületi elemek}

Ebben a fejezetben bemutatásra kerülnek a felhasználói felület elemei, melyeken
keresztül a felhasználó a programot használni fogja. A felhasználói felület
kialakítása során törekedtünk az egyszerűségre és az átláthatóságra, hogy a
felhasználó számára a lehető legegyszerűbb legyen a program használata. Ez
csökkenti a felhasználó által használt programidőt, ezáltal programunk kevésbé lesz
hajlamos a lassúságra, illetve az esetleges összeomlásra. Ezért az egyes funkciókat
úgy próbáltuk megoldani, hogy azok a felhasználó ezen a területen szerzett előzetes
tapasztalatainak megfeleljenek. A React.js alapvető bootstrapjét használjuk a
projekt implementálásánál, amelyet a következő parancs telepítése után vehetünk
igénybe:
\begin{python}
npm install react-bootstrap bootstrap 
\end{python}
majd beimportáljuk a projektbe. 
\begin{python}
<link
  rel="stylesheet" 
href="https://maxcdn.bootstrapcdn.com/bootstrap/4.5.0/css/bootstrap.min.css"
  integrity="sha384- 
    9aIt2nRpC12Uk9gS9baDl411NQApFmC26EwAOH8WgZ
    l5MYYxFfc+NcPb1dKGj7Sk" 
  crossorigin="anonymous" /> 
\end{python}	
Ezáltal képesek vagyunk reszponzív oldalt létrehozni, illetve felhasználni a
React JS \textit{Material} komponenseit, amelyeket a következő oldalon találhatunk: \\ \url{https://material-ui.com/components/snackbars/}.

\Section{Részlegek kezelése}

A részlegek felülete \aref{fig:reszlegek}. ábrán látható.

\begin{figure}[h!]
	\centering
	\includegraphics[width=\textwidth]{images/ListDepartment.png}
	\caption{Részlegek listázása}
	\label{fig:reszlegek}
\end{figure}

Vezetőként belépve, lehetőségünk van a részlegek megtekintésére, törlésére,
módosítására.

Az \textit{Edit} gomra kattintva, egy felugró ablak jelenik meg, amely kiolvassa, az aktuális
mezőhöz tartozó adatokat, ezek alapján lehet módosítani (\ref{fig:editreszleg}. ábra).

A felugró ablak hasonlóképpen, \aref{fig:editreszleg}. ábrán látható módon néz ki. (A mezőhöz tartozó \texttt{DepartmentID} inaktív, hiszen azt nem lehet módosítani.)

\begin{figure}[h!]
	\centering
	\includegraphics[width=11.8truecm, height=7.5truecm]{images/EditDepartment.png}\\
	\caption{Részleg szerkesztése}
	\label{fig:editreszleg}
\end{figure}

A törlés gomb megnyomásával egy \textit{Alert} rész ugrik fel, rákérdez arra, hogy
,,\textit{Biztosan szeretnéd törölni?}''.
Ugyanezen a formon lehetőségünk van egy részleg hozzáadására az \textit{Id} automatikusan inkrementálódik, tehát a felhasználónak csak magát a részleg
nevét kell megadnia.

\Section{Alkalmazottak adatainak kezelése}

Az Alkamazottak felülete \aref{fig:alkalmazottak}. ábrán látható.

\begin{figure}[h!]
	\centering
	\includegraphics[width=15.5truecm, height=6.5truecm]{images/ListEmployee.png}
	\caption{Alkalmazottak listája}
	\label{fig:alkalmazottak}
\end{figure}

A design és a funkciók hasonlóak az előző fórumom látottakhoz, a tárolt
adatok különböznek, illetve van még némi különbség.

Ha egy új dolgozót szeretnénk felvenni a rendszerbe, akkor a részleg
kiválasztásnál a meglévő részlegek közül tudunk választani egy legördülő
listában (\ref{fig:addEmployee}. ábra), ezzel korlátozva azt, hogy hibás részleget adjon meg a felhasználó
vagy nem létezőt.

\begin{figure}[h!]
	\centering
\includegraphics[width=14truecm, height=7.5truecm]{images/addemployee.png}\\
	\caption{Alkalmazott hozzáadásának a felülete}
	\label{fig:addEmployee}
\end{figure}

Az adatbázisban felvett dátum típus miatt problémák lehetnek azzal, ha
hibásan kezeljük az űrlapon implementált dátum mezőt.
Ebből kifolyólag egy dátum típusú \textit{Form Control} lett megadva a helytelen
konvenciók miatt.
Ez \aref{fig:editEmployeees}. ábrán tekinthető meg.

\begin{figure}[h!]
	\centering
	\includegraphics[width=10.8truecm, height=8.5truecm]{images/EditEmployee.png}
	\caption{Alkalmazott szerkesztésének a felülete}
	\label{fig:editEmployeees}
\end{figure}

\Section{Szerver és kliens közötti kommunikáció}

Mivel két különböző szerveren fut a Backend és a Frontend, így a kommunikációt JSON formátum segítségével oldhatjuk meg, amelyben a
Backend oldalon a következőkben bemutatásra kerülő részek szerepelnek az implementációban.

Ahhoz, hogy a két szerver kommunikálni tudjon egymással, és ne írjon hibát, fel
kell oldani egy CORS problémát a .NET résznél.
Mivel a frontend a 3000-es porton érhető el, így az implementációja az alábbi
formátumban lesz, a NuGet csomagok telepítése után:
\begin{python}
config.EnableCors(new
    EnableCorsAttribute("http://localhost:3000", "*", "*")); 
\end{python}

A JSON formátummá alakításhoz pedig a következő hívás szükséges:
\begin{python}
config.Formatters.JsonFormatter.SupportedMediaTypes.Add( 
    new MediaTypeHeaderValue("text/html")); 
\end{python}
Így, ha a fentebbi listát tekintjük meg Backend oldalon, akkor az alábbi JSON formátumot fogjuk látni (például a részleg esetén):
\begin{python}
{
    "$id": "1",
    "departmentId": 1005,
    "name": "Beszallitas",
    "leader": "John bill"
},
\end{python}%$
Az employee részleg esetén : 
\begin{python}
{
    "$id": "2",
    "employeeId": 1030,
    "fistName": "Steve",
    "lastName": "Jon",
    "adress": "Budapest",
    "phoneNumber": "06701112222",
    "email": "steve@gmail.com",
    "position": "Accountant",
    "deprartment": null,
    "departmentId": 1005,
    "tasks": null
},
\end{python}

A kliens oldali részről ahhoz, hogy ezt a JSON formátumot használhatóvá tudjuk
tenni, a \textit{ComponentDidMount} függvényen belül definiálni kell a következőket:
\begin{python}
fetch('https://localhost:44315/api/Department')
fetch('https://localhost:44315/api/Employee')
\end{python}

A modellezett üzleti folyamatot a felhasználó kétféleképpen tudja ábrázolni az alkalmazásban (\ref{fig:fetchapi}. ábra).

\begin{figure}[h!]
	\centering
	\includegraphics[width=\textwidth]{images/businessprocessDraw.png}\\
	\caption{Road Transportation Maintance ábrázolása az alkalmazásban.}
	\label{fig:fetchapi}
\end{figure}

\Aref{fig:roadtransporationdemand2ndabra}. ábrán látható ábrázolásban a lépések közötti sorrend, és az elágazások jobban átláthatóak.

\begin{figure}[h!]
	\centering
	\includegraphics[width=8truecm, height=12truecm]{images/2nd process diagram.png}\\
	\caption{Folyamatábra készítése a Road Transportation Maintance-hez.}
	\label{fig:roadtransporationdemand2ndabra}
\end{figure}

Amíg a kliens oldali rész nem került implementálásba, addig az alapvető CRUD
metódusok tesztelése a Postman segítségével történt (\ref{fig:postman}. ábra), ahol kiválaszthatjuk, hogy
POST / DELETE / UPDATE / DELETE metódust szeretnénk végrehajtani az adott URL-en.

\begin{figure}[h!]
	\centering
	\includegraphics[width=15.5truecm, height=4.5truecm]{images/PostmanJson.png}\\
	\caption{Postmanban történő tesztelés}
	\label{fig:postman}
\end{figure}

Itt egy adat kerül hozzáadásra az \textit{Alkalmazottak} mezőhöz. A POST metódusnál
a kivételkezelés kifejezetten fontos, így hibás adat esetén nem kaptunk volna sikeres visszajelzést.

Az új igénylő első lépése megtekinthető \aref{fig:newMaintance}. ábrán, az első jóváhagyó lépése pedig \aref{fig:newMaintance2}. ábrán.

\begin{figure}[hb!]
	\centering
		\includegraphics[width=15truecm, height=7truecm]{images/newRoad1.png}\\
	\caption{Az új környezetben lévő Road Transportation Maintenance}
	\label{fig:newMaintance}
\end{figure}

\begin{figure}[hb!]
	\centering
		\includegraphics[width=15truecm, height=7truecm]{images/newRoad2.png}\\
	\caption{Az új környezetben lévő Maintenance Approve lépés}
	\label{fig:newMaintance2}
\end{figure}

Az ehhez tartozó implementáció egy részlete az alábbi.
\begin{python}
componentDidMount() {
  this.refreshList();
}

refreshList() {
  fetch('https://localhost:44315/api/Draft')
    .then(response => response.json())
    .then(data => { 
      this.setState({draft:data});
    }
  );
}
\end{python}

Telefonon vagy egyéb kisebb eszközön az űrlap \aref{fig:responsive}. ábrán látható formában kerül megjelenítésre. 

\begin{figure}[h!]
	\centering
	\includegraphics[width=7.5truecm, height=14truecm]{images/Responsive.png}
	\caption{Az igénylő fórum reszponzív megjelenése.}
	\label{fig:responsive}
\end{figure}
