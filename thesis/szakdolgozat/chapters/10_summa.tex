\chapter{Összefoglalás}

A szakdolgozatomhoz elkészített szoftver implementálása során részletesen megismertem napjaink népszerű technológiáit és keretrendszerét, a React-ot és a .NET keretrendszert.
A webalkalmazásban elkészített funkciók során egy React-hoz elérhető folyamatábrák megjelenítésével kapcsolatos könyvtárat ismertem meg. Továbbá az autentikáció implementálását és 
reszponzív űrlapok létrehozását is megtanultam.

Szerver oldalon a táblák elkészítését, lekérdezését, és összekapcsolását, az Entity Framework használatával végeztem. A dolgozatban tehát az alkalmazás elkészítése során a teljes webalkalmazás stack-et láthatjuk. Az alkalmazás elkészítése során a két modellezett üzleti folyamat űrlapjai elkészítésre kerültek, és további olyan hasznos funkciók, amelyek a vállalat irányítását megkönnyíthetik. Az alkalmazottaknak bejelentkezés, vagy regisztráció után van lehetősége használni az alkalmazást.  

További fejlesztési lehetőség az alkalmazás elkészítésében a különböző jogosultsági 
szintek megkülönböztetése azért, hogy elkülöníti a felhasználókat a vezetőktől, ezáltal a feladat küldése is megvalósítható.

A szakdolgozat elméleti részében részletesen bemutatásra kerültek a különböző előforduló és felhasznált folyamatmodellek. Ezeket különböző matematikai módszerek segítségével vizsgáltuk, annak érdekében, hogy jobban belelássunk az előnyeikbe és a hátrányaikba.

A folyamatmodellezés vizsgálatainál sok segítséget jelentettek a korábban már egyetemen megtanult valószinűségszámítási és optimalizálási ismeretek.
Az elméleti rész során felhasználásra került sok korábban tanult összefüggés, képlet és kritérium.

A szakdolgozat részletesen taglalja, hogy hogyan lehet optimális megoldást találni egy adott problémára.
Továbbá az automatákkal kapcsolatos fejezetnél hasznos volt, az automaták és formális nyelvek kurzuson tanult definíciók és összefüggések.

A dolgozatban áttekintést kaphatunk a Petri hálók működéséről és használatukról, továbbá a nem determinisztikus automatákról és a nem struktúrált folyamatgráfokról.
 
A webalkalmazásban elkészített folyamatábrák segítségével tetszőleges üzleti folyamatokat tudunk elkészíteni. Ezeknél egy további érdekes fejlesztési irány lenne a strukturáltság automatikus ellenőrzése.
