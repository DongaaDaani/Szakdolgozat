\Chapter{Modellezett üzleti folyamatok}

A fejezet áttekinti a modellezett üzleti folyamatokat és az azokhoz tartozó grafikus elemeket.

\Section{Road Transportation Demand}

Az első üzleti folyamat, amely a személygépkocsi igénylése, tulajdonképpen egy
elég egyszerű (2 lépéses folyamat). Ennek a fejlesztésénél talán a felugró mezők okozzák a
legnagyobb kihívást.

A folyamat első, \textit{draft} lépése \aref{fig:draft}. ábrán látható módon néz ki.

\begin{figure}[h!]
	\centering
	\includegraphics[width=\textwidth]{images/road1.png}
	\caption{Road Transportation Demand Draft Page}
	\label{fig:draft}
\end{figure}

Az igénylés típusát kiválasztva, olyan mezők adódnak hozzá a folyamathoz, amelyek a jelenlegi
képernyő fotón nem látszódnak.

A következő az igénylő szervezeti egység lenne, amely egy felugró kereső ablak, ahol a felhasználó
kiválasztja azt, hogy melyik szervezethez tartozik.

Ennek adatai egy külön adatbázis táblában kapnának helyet, amelyben a cégnél nyilvántartott szervezetek lesznek eltárolva,
majd a felugró ablakban, ezekből lehet választani.

Több mező is a felhasználó álltál kiválasztott Checkbox szerint jelenik majd meg, és tűnik el.
Vegyük példaként azt, ha a felhasználó kiválasztja az igénylés típusánál a „Járművezetővel ellátott
személygépkocsit” (\ref{fig:roadtransportationdemand}. ábra).

\begin{figure}[h!]
	\centering
	\includegraphics[width=\textwidth]{images/road2.png}
	\caption{Road Transportation Demand Draft Page}
	\label{fig:roadtransportationdemand}
\end{figure}

Ekkor megjelenik egy új kiválasztós mező, amelyben az \textit{Oda út és visszaút}ról kérdez, illetve négy
további mező, amely a gépjármű elvitelének és a gépjármű leadásának az idejéről, illetve a gépjármű
elviteli ideje és leadási ideje.

Ha csak „egy út” van kiválasztva, akkor a gépjármű leadásának dátuma, és ideje mező eltűnik,
ellenkező esetben megmarad.

Lehetőségünk van több utast is hozzáadni, amelyben a hozzáadás gomra kattintva felugrik egy rész,
ahol az adatokat kérjük le.
Ez szintén egy új táblázat lesz (Road Transportation Demand DTL), amelyből 1 folyamathoz, akár több
is tartozhat.

\begin{figure}[h!]
	\centering
	\includegraphics[width=\textwidth]{images/road3.png}
	\caption{Felugró ablak}
	\label{fig:popUp}
\end{figure}

Itt az \textit{Utas neve}, egy szintén felugró ablak, amelyből kiválaszthat a cég alkalmazottak közül egyet (\ref{fig:popUp}. ábra).

Az utasok a \textit{User} táblából lesznek Listázva, amelyben nyilván van tartva a cég összes alkalmazottja,
annyi eltéréssel, hogy az utasnak nem lesz költség helyi kódja.

Tetszőleges számú utast adhatunk hozzá, illetve tudjuk törölni is őket.
Ha kitöltötte a felhasználó a mezőket, akkor a \textit{Submit} gombra kattintva a következő, approve lépésbe
ér a folyamat, ahol a meglévő adatok kiolvasása történik. Itt láthatja a következő illetékes azt, hogy
milyen adatokkal szeretnének járművel bérelni.

Mivel ez egy egyszerű, 2 lépéses folyamat, egy Draft és egy Approve lépésből állt, a folyamat ezt
követően véget is ért.

\Section{Road Transportation Maintance}

Második üzleti folyamat, amely egy karbantartási folyamat (\ref{fig:maintranceDraft}. ábra).
Ez egy komplexebb folyamat, itt több lépés van, mint az előzőben, illetve itt is
megtalálhatóak a \textit{Checkbox} szerinti kiválasztással felugró mezők.
Az első lépést, az-az a \textit{draft}-ot itt láthatjuk \aref{fig:maintranceDraft}. ábrán.

\begin{figure}[h!]
	\centering
	\includegraphics[width=\textwidth]{images/road4.png}
	\caption{Maintance draft page}
	\label{fig:maintranceDraft}
\end{figure}

\noindent Ebben az első kiválasztott érték esetén, ha az nem jármű, akkor az alábbi mezőt kapjuk meg (\ref{fig:realestate}. ábra).

\begin{figure}[h!]
	\centering
	\includegraphics[width=\textwidth]{images/road5.png}
	\caption{Real Estate checked}
	\label{fig:realestate}
\end{figure}

Amennyiben a folyamatban a Jármű kerül kiválasztásra, akkor egy rendszámot kell beírni, amely
szintén egy külön tábla lesz, hiszen több rendszámot kell nyilvántartani, és a gépjárművekhez tartozó
további adatokat is, mivel a keresés gomra kattintva, automatikusan kitöltődnek a csak olvasható (\textit{read only}) mezők
értékei, mint a Márka, típus, Üzemóra, Futott Kilométer stb.

Ezután a probléma leírása rész következik, ahol egy tetszőleges leírást adhat az igénylő felhasználó.
Ezt követően a második (\textit{Approve}) lépésben az alábbi rész jelenik meg (\ref{fig:maintanceSupervisor}. ábra).

\begin{figure}[h!]
	\centering
	\includegraphics[width=\textwidth]{images/road6.png}
	\caption{Road Transportation Maintance Supervisor Approve}
	\label{fig:maintanceSupervisor}
\end{figure}

Itt a kiolvasott adatok mellett megjelenik egy újabb kitöltő mező, amelyben a \textit{Supervisor Description}-t
kell megadni.

A következő 3. lépésben azt kell megadni, hogy mi a probléma típusa, amely lehet Személyes,
Belső, vagy külső probléma (\ref{fig:coordinatorApprove}. ábra).

Amennyiben a probléma típusa Belső, vagy Külső, akkor egy újabb mező ugrik fel, amelyben szintén a
meglévő alkalmazottak közül kell kiválasztani azt, hogy a folyamat kihez megy tovább, ki lesz érte a
felelős. A megjelenő oldalt \aref{fig:coordinatorApprove}. ábrán láthatjuk.
 
\begin{figure}[h!]
	\centering
	\includegraphics[width=\textwidth]{images/road7.png}
	\caption{Road Transportation Maintance Coordinator Approve}
	\label{fig:coordinatorApprove}
\end{figure}

Tesztelésként kiválasztjuk az \textit{Internal}-t, majd továbbítjuk, akkor az általunk megadott \textit{External} szerint fog a folyamat tovább menni (\ref{fig:mechanicalApprove}. ábra).

\begin{figure}[h!]
	\centering
	\includegraphics[width=\textwidth]{images/road8.png}
	\caption{Road Transportation Maintance Coordinator Approve}
	\label{fig:mechanicalApprove}
\end{figure}

Ezután, az illetékes az alábbi formot fogja látni a felhasználó (\ref{fig:laststepmaintanceapprove}. ábra).

\begin{figure}[h!]
	\centering
		\includegraphics[width=\textwidth]{images/road9.png}\\
	\caption{Road Transportation Maintance utolsó lépése}
	\label{fig:laststepmaintanceapprove}
\end{figure}

\noindent Itt az alábbi adatokat megadva, visszatér a folyamat az előző lépésben lévő emberhez, aki megkapja
ezeket az adatokat, és jóváhagyja.

Ezáltal vége a folyamatnak. Amennyiben a „\textit{Personal}” részt választotta volna ki, akkor ott lett volna
vége a folyamatnak.

