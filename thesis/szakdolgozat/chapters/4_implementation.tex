\Chapter{Üzleti folyamatok}

Egy vállalat élete során rengeteg problémával néz szembe, akár napi szinten. Ezek lehetnek egyszerű, vagy teljesen komplex feladatok is. Ahhoz, hogy egy vállalat zökkenőmentesen működjön, fontos ezeket a feladatokat könnyedén, hatékonyan megoldani. A fejezet ennek lehetséges módjaival foglalkozik.

\Section{Workflow rendszerek}

Üzleti folyamatoknál gyakran emlegetik a {\bf workflow} (munkafolyamat) kifejezést, amelynek a jelentése olyan tevékenységek sorozata, amelyeket az gépek vagy az emberek végrehajtanak egy logikai terv szerint valamilyen cél elérése érdekében. 
Magában foglalja az üzleti folyamatok definiálását, végrehajtását és automatizálását, ahol a feladatok, információ vagy dokumentumok kerülnek átadásra egy résztvevőtől egy másikhoz az eljárás szabályainak
megfelelően.

% TODO: A BPM rövidítést ki kellene írni, és be kellene hivatkozni. (Más, hasonló hivatkozások is jól jöhetnek ide.) 

A korábbi workflow rendszerek csak az üzleti folyamatok leképezését tették lehetővé vizualizációs eszközök segítségével. A Business Process Management azóta sokat fejlődött és a támogató szoftvercsomagok sokkal több lehetőséget nyújtanak már.
Ilyenek lehetnek a \textit{Process Designer}-be integrált plusz funkciók, mint a folyamatokhoz tartozó folyamatábra, a folyamat lépéseihez tartozó űrlapok megtekintése, és az, hogy az egyes mezők mögött milyen metódushívások, és adatbázis van. Az összes üzleti folyamatot egy helyen lehet tárolni
\cite{Bene}.

\Section{Elvárások}

Egy folyamatmodellezőnek a következő képességekkel kell rendelkezniük \cite{Bene}:
\begin{itemize}
\item Mérések: Lehetőséget adnak arra, hogy a folyamatokat valós időben átláthassuk.
\item Modellezés: Workflow-k és feladatok tervezése, informatikai eszközökkel.
\item Portálok: Egységes felhasználói felületek, webes felület végfelhasználók számára.
\item Mobilitás: Reszponzív felület, bármilyen képernyőn nyomon követhető.
\item Metaadatok : Leírja a folyamatokhoz tartozó jellemezőket.
\item Szimuláció: A folyamatmodell alkalmazásával különféle forgatókönyv és input adatok szerinti vizsgálata a munkafolyamatnak.
\item Analízis: Elemzések, jelentések.
\end{itemize}

\Section{Folyamatgráf}

A folyamatgráf (\textit{Process Graph}, \textit{P-Graph}) egyik nagy előnye a folyamatok gráf alapú megközelítésnek, mivel a gráfelméleti algoritmusok alkalmazhatóak a vizsgálat során.

A folyamatgráf, egy speciális irányított páros gráf optimalizálási feladatok hatékony megoldásához, vagy épp
workflow modellezéshez \cite{Bene}.

\Section{Folyamatszintézis}

A nagy szoftverek kisebb részekből állnak. A kisebb részek szintén úgy épülnek fel, hogy még kisebb programrészeket használnak fel. Egy-egy programrészről tudjuk, hogy mely adathalmazt és objektum összességet használja és ezekből milyen adatokat állít elő.
A tervezés fontos része ezen komponensek összeillesztése, illetve összehozása.

\Section{Folyamatmodellezés és szimuláció}

% TODO: A nemdeterminisztikus modelleknek milyen jelentősége lesz a későbbiekben? (Nem teljesen tünne indokoltnak ilyen formában a témához.) (A Későbbiekben nem lesz jelentősége, viszont az első két folyamat nem determinisztukus, hiszen két bemente van. Talán azt itt ki lehet hangsúlyozni.)

\Aref{fig:Demand_Process}. és \aref{fig:maintane_process} ábrán látható folyamatábra nem determinisztikus volt, mivel mindkettőnél két bemenet volt; az egyik a \textit{Draft}, a másik pedig a \textit{Start} lépés. A \textit{Start} lépést a \textit{Draft} lépés követi.

A legtöbb korszerű workflow rendszer főbb szolgáltatásai alkalmasak a munkafolyamatok között eltelt idő, várakozási és sorbanállási idők csökkentésére, és minimalizálására. A sorbanállási modell nem elhanyagolható, mivel gyakran felhasználják üzleti döntéseknél, ahol az erőforrások becslése nem elhanyagolható \cite{sorbanallaswikipedia}. Könnyen előfordulhat ugyanis nagyobb mértékű feltolódás egy bizonyos erőforráson, amely nagy kapacitást igényel. A sorbanállási modell a hétköznapi életben is előforduló egyik kellemetlen jelenség a
várakozás vizsgálatával foglalkozik. Definíció szerint a sorbanállás-elmélet különböző
folyamatok eseményeivel kapcsolatos várakozási sorokat, sorbanállási időket a
kiszolgálásra, és ezek összefüggéseit tárgyalja az alkalmazott matematika eszközeivel \cite{Sorba}.

A folyamat során várakozó sor keletkezhet, ha a kiszolgáló egységekbe történő áramlás
időköze, és a kiszolgálás időtartama szabálytalan. Sorbanállás keletkezhet akkor is, a
beáramlás időköze kisebb, mint a kiszolgálás időtartama. Ekkor a tároló térben a beáramló
várakozó anyagmennyiség, azaz várakozó sor folyamatosan növekszik.

A gyakorlati életben a beérkezés időköze, és a kiszolgálás ideje nem meghatározott, hanem
valószínűségi változó. Ekkor sztochasztikus folyamatról beszélünk, amelynek megoldása
bizonyos feltételek mellett analitikusan végrehajtható. A szakirodalom alapján nézzük meg ezek fő definícióit, összefüggéseit és az azokhoz kapcsolódó definíciókat \cite{istvan2009valoszinHusegszamitas}.

A várakozó sorba időegység alatt
beérkező egységek száma, mint valószínűségi változó leggyakrabban Poisson-féle eloszlást
követ, amely egy diszkrét valószínűségi változó, amelynek a definíciója a következő.

Legyen $\lambda > 0$ egy rögzített valós szám. Azt mondjuk, hogy a $\xi$ valószínűségi változó Poisson
-eloszlású $\lambda$ paraméterrel, ha eloszlása:
\begin{equation}
	p_{k} = P(\xi = k) = { {\lambda^{k}}\over{k!} } \cdot e^{-\lambda}.
\end{equation}

A $\xi$ várható értéke és szórásnégyzete $E(\xi) = D^{2}(\xi)=\lambda$. A várható érték képlete diszkrét esetben :
\begin{align*}
	\sum_{i=1}^{n} (x_{i}-a) \cdot p_{i} & = 0, \\
	\sum_{i=1}^{n} x_{i} \cdot p_{i} - a* \sum_{i=1}^{n} p_{i} & = 0, \\
	a = { \sum_{i=1}^{n} x_{i} \cdot p_{i} \over{\sum_{i=1}^{n} p_{i}} }, \\
	\sum_{i=1}^{n} p_{i} = 1 \text{ ebből következik } a & = \sum_{i=1}^{n} x_{i} \cdot p_{i}.
\end{align*}
Fontos megjegyezni, hogy az $x_{1}, x_{2}, \dots $  $\exists E(\xi) = \sum_{i=1}^{\infty} x_{i} \cdot p_{i}$, ha $\sum_{i=1}^{\infty} \mid x_{i} \cdot p_{i} \mid < + \infty$.

A Poisson-eloszlás várható értékének a levezetése:

\begin{align*}
	E(\xi) = \sum_{k=0}^{\infty} x_{k} \cdot p(k,\lambda)  = \sum_{k=0}^{\infty} k {\lambda^{k} \over{k!}} \cdot  e^{-\lambda} = \\
	\sum_{k=1}^{\infty} { \lambda^{k} \over{ (k-1)! } } \cdot  e^{-\lambda} = \lambda \cdot e^{-\lambda} \sum_{k=1}^{\infty} { \lambda^{k-1} \over{ (k-1)!}}
	= & \lambda \cdot e^{-\lambda} \cdot e^{\lambda} = \lambda.
\end{align*}

A Poisson-eloszlás szórás négyzete $ D^{2}(\xi) = \lambda $ levezetéséhez felhasználjuk a \\ $D^{2}(\xi) = E(\xi^{2})-E^{2}(\xi)$ összefüggést. Ez a második centrális momentum alapján könnyen igazolható, hiszen:
\[
E( (\xi - E(\xi)^2 ) = E(\xi^2 - 2\xi E(\xi) + E^2(\xi) ).
\]
Itt a véges additivitást követően az alábbi képlet adódik:
\[
E(\xi^2) - 2*E(\xi)*E(\xi) + E^2(\xi) =  E(\xi^{2})-E^{2}(\xi)
\]
Ezeket az összefüggéseket felhasználva belátható, hogy:
\begin{align*}
	E(\xi^2)   = \sum_{k=0}^{} k^{2} \cdot { \lambda^k \over{k!} }   e^{-\lambda}  = \sum_{k=1}^{\infty} [(k-1)+1]  \\ { \lambda^k \over{(k-1)!} } \cdot e^{-\lambda}  = \sum_{k=2}^{\infty} (k-1){ \lambda^k \over{ (k-1) !} } \cdot e^{-\lambda}   + \\ \sum_{k=1}^{\infty} { \lambda^k \over{(k-1)!} } \cdot e^{-\lambda}   =  \lambda^2 e^{-\lambda} \sum_{k=2}^{\infty} { \lambda^{k-2} \over{ (k-2) ! } } + \lambda e^{-\lambda} \sum_{k(1}^{\infty} { \lambda^{k-1} \over{(k-1)!} } = \lambda^2 + \lambda
\end{align*}
A képletbe helyettesítve adódik, hogy
\[
D^{2} (\xi) =  E(\xi^{2})-E^{2}(\xi) =(\lambda^2 + \lambda) - \lambda^2 = \lambda.
\]

\SubSection{A Poisson eloszlás és exponenciális eloszlás kapcsolata}

Ha időegység alatt bekövetkező események száma Poisson eloszlású
valószínűségi változó, akkor két egymást követő bekövetkezés között eltelt idő exponenciális eloszlású ugyanazzal a $\lambda$ paraméterrel. Az exponenciális eloszlás egy folytonos valószinűségi változó. 

Legyen a $\xi$ abszolút folytonos valószínűségi változó sűrűségfüggvénye $f_{\xi}(x)$. Ekkor a $\xi$ várható értéke
\begin{equation*}
E(\xi) = \int_{-\infty}^{\infty} x f_{\xi}(x) dx,
\end{equation*}
ha az integrál konvergens, azaz $\int_{-\infty}^{\infty} \mid x \mid f_{\xi}(x) < \infty$.

Az $F(x) = P(\xi < x) $ formulával meghatározott valós függvényt a $\xi$ valószínűségi változó eloszlásfüggvényének nevezzük. \\

A sűrűségfüggvény definíciója a következő.
Ha létezik egy $f$ nemnegatív valós függvény, amelyre
\begin{equation*}
F(x) = \int_{-\infty}^{x} f(t) dt \qquad \forall x \in \Re
\end{equation*}
akkor az $f$ az $F$ eloszlásfüggvényhez tartozó sűrűségfüggvény.

A $\xi$ valószínűségi változót $\lambda$ paraméterű exponenciális eloszlásúnak nevezzük, ha eloszlásfüggvénye:
\begin{align*}
	F(x) & = 0 ,\qquad x \leq 0, \\
	F(x) & = 1-e^{-\lambda x} \qquad x > 0.
\end{align*}
Az $f(x) = F'(x)$ definíció alapján az exponenciális eloszlás sűrűségfüggvénye:
\begin{align*}
	f(x) & = 0, \qquad x \leq 0, \\
	f(x) & = \lambda e^{-\lambda x} \qquad x > 0.
\end{align*}

A folyamatos valószínűségi változó várható értéke az $E(\xi) = \int_{-\infty}^{\infty} x f(\xi) dx$ alapján:
\begin{align*}
	f_{\xi}(x) & = \lambda e^{-\lambda x }, \\
	\int_{-\infty}^{\infty} f_{xi} (x) & = 0, \\
	E(\xi) = & \int_{-\infty}^{\infty} x \lambda e^{-\lambda x} dx = \\
	& x e^{-\lambda x} - \int_{0}^{\infty} 1 e^{-\lambda x} dx
\end{align*}
Ha megvizsgáljuk az integrál előtti részt, $x = 0$ és $ x = \infty$ értéket behelyettesítve, akkor annak az értéke nulla.
Marad tehát a $-\int_{0}^{\infty} 1 e^{-\lambda x} dx$ rész.
\begin{equation*}
	-{1\over{\lambda}} * \int_{0}^{\infty} \lambda e^{-\lambda x} \text{ ami a sűrűségfüggvény}.
\end{equation*}

Az alap definíció alapján $\int_{\infty}^{\infty} f_{xi}(x) dx = 1$, ebből kifolyólag $E(\xi) = {1\over{\lambda}}$.

A $D(\xi) = \sqrt{E(\xi ^2) - E^{2}(\xi)}$ szórás képletét felhasználva, ahol $E^{2}(\xi) = {1\over{\lambda^2}}$, az alábbi képlet adódik:
\begin{align*}
	& E(\xi^{2}) = \int_{-\infty}^{\infty} x^{2} f_{\xi}(x) dx = x^{2}* e^{-\lambda x } - \int_{0}^{\infty} 2x e^{-\lambda x} dx ={ 2 e^{-\lambda x} \over{\lambda^2} } = { 2 \over{\lambda^2} } \\
	& = \sqrt{ {2 \over{\lambda^2} }  - {1 \over{\lambda^2} } } =  {1 \over{\lambda} }.
\end{align*}
Így a $D^{2}(\xi) = {1\over{\lambda^2}}$. (A bizonyítás a karakterisztikus függvény segítségével is levezethető.)

\Section{Szabványosított folyamatleíró nyelvek}

A következő pontokban néhány elterjedt, szabványosított folyamatleírási módot tekintünk át.

\SubSection{IDEF diagramok}

Az \textit{Integrated Definition} (IDEF) folyamatleíró nyelvet elsősorban folyamatok
fejlesztéséhez, integrációjához, tervezéséhez és rendszerelemzéshez kapcsolódó tevékenységek leírására használják \cite{Bohacs}.

A modellkészítés első lépése, a
rendszer alapvető funkciói, tevékenységei közötti kapcsolati feltételeket állapítja meg.
Ezen felül részletes leírást ad a rendszer folyamatairól, tevékenységeiről.
Részei \cite{Bene}:
\begin{itemize}
\item  \textbf{IDEF0}: Funkciómodellezés,
\item \textbf{IDEF1}: Információs modellezés,
\item \textbf{IDEF1X}: Adatmodellezés,
\item \textbf{IDEF2}: Szimulációs modelltervezés,
\item \textbf{IDEF3}: Folyamatleírás rögzítése,
\item \textbf{IDEF4}: Objektum-orientált tervezés - Az objektumorientált programozás, karbantarthatóság és kód újrafelhasználhatóság megvalósítható a hagyományos adatfeldolgozó alkalmazásokban.
\end{itemize}

\SubSection{Event Driven Process Chain}

Előfordulhat, hogy ugyanazt az üzleti folyamatot egyszerre több ember vagy gép végzi. Akár több független igénylés is érkezhet, ami több összetett lépésből áll. Ez lehet több egyszerre történő módosítás egy nagy gyárban. Ilyenkor meg kell különböztetni ezeket a folyamatokat, párhuzamosan kell tudni kezelni.
Erre szolgál az EPC (\textit{Event Driven Process Chain}), amely az események és funkciók irányított gráfja.
Különböző logikai kapcsolatok
hozhatók benne létre, amelynek révén alkalmas alternatív vagy párhuzamos lefutású
folyamatok modellezésére. Ezekhez olyan logikai operátorokat használ, mint az OR, AND
és a XOR.

XOR, vagy más néven "kizáró vagy" az alábbi formulával adható meg: $ ( \bar{x} \land y  ) \lor ( x \land \bar{y} )$.
Az EPC felhasználható a vállalati erőforrás-tervezés végrehajtásának konfigurálására és az üzleti folyamatok fejlesztésére.


\SubSection{Yet Another Workflow Language}

A YAWL (\textit{Yet Another Workflow Language}) Elsősorban a munkafolyamatokban előforduló tevékenységekre koncentrált, viszont mára
 munkafolyamat leírásának teljes eszköztárát támogatja és segíti. \cite{Bohacs}.

A nyelvet egy olyan szoftverrendszer támogatja, amely egy végrehajtó motort, egy grafikus szerkesztőt és egy munkalista kezelőt tartalmaz.

A nyelvet és támogató rendszerét eredetileg az Eindhoveni Műszaki Egyetem és a Queenslandi Műszaki Egyetem kutatói fejlesztették ki .

Főbb jellemzői:
\begin{itemize}
	\item A munkafolyamatok átfogó támogatása.
	\item A fejlett erőforrás-elosztási részek támogatása, beleértve a négy szem elvét és a láncolt végrehajtást.
	\item A munkafolyamat-modellek dinamikus adaptálásának támogatása. \\
	Kifinomult munkafolyamat-modell érvényesítési funkciók (Holtpont észlelés a tervezési időben).
\end{itemize}

\SubSection{XML Process Definition Language}

Az XPDL (\textit{XML Process Definition Language}) a WfMC (\textit{Workflow Management Coalition})
által szabványosított leírónyelv. Az XPDL egy olyan XML sémát definiál, ami alkalmas a
különböző leíró és modellezési eszközök közötti információcserére. Fontos kiemelni, hogy
az XPDL a folyamat leírása mellett a folyamat végrehajtására is tartalmaz információkat \cite{Wikixpdl}.

\SubSection{SysML}

A SysML egy általános célú leíró nyelv mérnöki alkalmazásokra kifejlesztve. Specifikációk,
analízis, tervezés, felülvizsgálat és jóváhagyás területén nyújt támogatást. A SysML
tulajdonképpen az UML kiterjesztett változata.

A folyamatok, algoritmusok leírását a mérnöki/műszaki tudományok területén kezdték
formalizálni az 1930-as években \cite{Bohacs}.

