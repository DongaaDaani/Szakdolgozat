\Chapter{Modellezett üzleti folyamatok}
\section{Road Transportation Demand}
Az első üzleti folyamat, amely a személygépkocsi igénylése, tulajdonképpen egy
elég egyszerű (2 lépéses folyamat), itt talán felugró mezők okozzák a
legnagyobb kihívást.\\
A folyamat első, draft lépése így nézne ki :   \ref{fig:draft} \\
{\large
	\begin{center}
		\vglue 1truecm
		\textbf{\huge\textsc{Road Transportation Demand első lépése}}\\
		\vglue 1truecm
		\includegraphics[width=11.8truecm, height=7.5truecm]{images/road1.png}\\
		\textbf{\textsc{Draft page}}
		\label{fig:draft}
\end{center}}

Az igénylés típusát kiválasztva, olyan mezők adódnak hozzá a folyamathoz, amelyek a jelenlegi
képernyő fotón nem látszódnak. \\
A következő az igénylő szervezeti egység lenne, amely egy felugró kereső ablak , ahol a felhasználó
kiválasztja azt, hogy melyik szervezethez tartozik. \\
Ez egy külön adatbázis tábla lenne, amelyben a cégnél nyilvántartott szervezetek lesznek eltárolva,
majd a felugró ablakban, ezekből lehet választani. \\
Több mező is a felhasználó álltál kiválasztott Checkbox szerint jelenik majd meg, és tűnik el.
Vegyük példaként azt, ha a felhasználó kiválasztja az igénylés típusánál a „Járművezetővel ellátott
személygépkocsit” : \ref{fig:roadtransportationdemand} \\



\begin{figure}[h]
	\centering
		\includegraphics[width=11.8truecm, height=7.5truecm]{images/road2.png} \\
	\caption{Road Transportation Demand Draft Page}
	\label{fig:roadtransportationdemand}
\end{figure}


Ekkor megjelenik egy új kiválasztós mező, amelyben az Oda út és visszaútról kérdez, illetve négy
további mező, amely a gépjármű elvitelének és a gépjármű leadásának az idejéről, illetve a gépjármű
elviteli ideje + leadási ideje. \\
Ha csak „egy út” van kiválasztva, akkor a gépjármű leadásának dátuma, és ideje mező eltűnik,
ellenkező esetben megmarad. \\
Lehetőségünk van több utast is hozzáadni, amelyben a hozzáadás gomra kattintva felugrik egy rész,
ahol az adatokat kérjük le. \\
Ez szintén egy új táblázat lesz (Road Transportation Demand DTL), amelyből 1 folyamathoz, akár több
is tartozhat. \\

\begin{figure}[h]
	\centering
	\includegraphics[width=11.8truecm, height=7truecm]{images/road3.png}
	\caption{Felugró ablak}
	\label{fig:popUp}
\end{figure}


Itt az Utas neve, egy szintén felugró ablak, amelyből kiválaszthat a cég alkalmazottak közül egyet. \ref{fig:popUp} \\
Az utasok a User táblából lesznek Listázva, amelyben nyilván van tartva a cég összes alkalmazottja,
annyi eltéréssel, hogy az utasnak nem lesz költség helyi kódja. \\
Tetszőleges számú utast adhatunk hozzá, illetve tudjuk törölni is őket. \\
Ha kitöltötte a felhasználó a mezőket, akkor a Submit gombra kattintva a következő, approve lépésbe
ér a folyamat, ahol a meglévő adatok kiolvasása történik. Itt láthatja a következő illetékes azt, hogy
milyen adatokkal szeretnének járművel bérelni. \\
Mivel ez egy egyszerű, 2 lépéses folyamat, egy Draft és egy Approve lépésből állt, a folyamat ezt
követően véget is ért. \\

\section{Road Transportation Maintance}

 második üzleti folyamat, amely egy karbantartási folyamat. \ref{fig:maintranceDraft} \\
Ez egy komplexebb folyamat, itt több lépés van, mint az előzőben, illetve itt is
megtalálhatóak a Checkbox szerinti kiválasztással felugró mezők. \\
Az első lépést, az-az a draft-ot itt láthatjuk : \\
\begin{figure}[h]
	\centering
	\includegraphics[width=11.8truecm, height=4.5truecm]{images/road4.png}
	\caption{Maintance draft page}
	\label{fig:maintranceDraft}
\end{figure}


\begin{python}
	<el-row :gutter="20">
	<el-col :span="12">
	<el-form-item :label="$t('BCSMRTMI18.demandType')" 
	prop="demandType">
	<el-radio-group v-model=BCSMRTM.whBtBcRtm.demandType"
	 @change="onVegicleChange">
	<el-radio v-for="item in BCSMRTM.selectList-demandType" 
	:key="item.relationKey" :label="item.relationKey">
	{{item.value_}}
	 </el-radio>
	</el-radio-group>
	</el-form-item>
	</el-col>
	</el-row>	
	
	<el-row :gutter="20" v-if="vehicleFieldVisible===true">
	<el-col :span="12">
	<el-form-item :label="$t('BCSMRTMI18.lplate')" 
	prop="lplate">
	<el-input type="text" v-model="BCSMRTM.whBtBcRtm.lplate"
	 placeholder="XXX-123">
	<el-button slot="append" @click="licensePlateClick()" 
	style="width:40px" size="small icon="el-search-icon"> 
	</el-button>
	</el-input>
	</el-form-item>
	</el-col>
	<el-col :span=12>
	<el-form-item :label="$t('BCSMRTMI18.brand')" 
	prop="brand">
	<el-input type="text" v-model="BCSMRTM.whBtBcRtm.brand" readonly>
	</el-input>
	</el-form-item> 
	</el-col>
	</el-row>

	<el-row :gutter="20" v-if="vehicleFieldVisible===true">
	<el-col :span=12>
	<el-form-item :label="$t('BCSMRTMI18.type')" 
	prop="type">
	<el-input type="text" v-model="BCSMRTM.whBtBcRtm.type"
	 readonly>
	</el-input>
	</el-form-item> 
	</el-col>
	<el-col :span=12>
	<el-form-item :label="$t('BCSMRTMI18.model')" 
	prop="model">
	<el-input type="text" v-model="BCSMRTM.whBtBcRtm.model"
	 readonly>
	</el-input>
	</el-form-item> 
	</el-col>
	</el-row>
	
	<el-row :gutter="20" v-if="vehicleFieldVisible===true">
	<el-col :span=12>
	<el-form-item :label="$t('BCSMRTMI18.totalMileage')"
	 prop="totalMileage">
	<el-input type="text" v-model="BCSMRTM.whBtBcRtm.totalMileage"
	 readonly></el-input>
	</el-form-item> 
	</el-col>
	<el-col :span=12>
	<el-form-item :label="$t('BCSMRTMI18.totalUsage')"
	 prop="totalUsage">
	<el-input type="text" v-model="BCSMRTM.whBtBcRtm.totalUsage"
	 readonly></el-input>
	</el-form-item> 
	</el-col>
	</el-row>
	
	<el-row :gutter="20>
	<el-col :span="24">
	<el-form-item :label="$t('BCSMRTMI18.propertyDesc')" 
	prop="propertyDesc">
	<el-input type="textare"
	 v-model="BCSMRTM.whBtBcRtm.propertyDesc">
	</el-input>
	</el-form-item> 
	</el-col>
	</el-row>
	
\end{python}

Itt az első kiválasztott érték esetén, ha az nem jármű, akkor az alábbi mezőt kapjuk meg: \ref{fig:realestate} \\


\begin{figure}[h]
	\centering
	\includegraphics[width=15.5truecm, height=2.7truecm]{images/road5.png} \\
	\caption{Real Estate checked}
	\label{fig:realestate}
\end{figure}

Amennyiben a folyamatban a Jármű kerül kiválasztásra, akkor egy rendszámot kell beírni, amely
szintén egy külön tábla lesz, hiszen több rendszámot kell nyilvántartani, és a gépjárművekhez tartozó
további adatokat is, mivel a keresés gomra kattintva, automatikusan kitöltődnek a readonly mezők
értékei, mint a Márka, típus, Üzemóra , Futott Kilométer stb. \\
Ezután a probléma leírása rész következik, ahol egy tetszőleges leírást adhat az igénylő felhasználó.
Ezt követően a második (Approve) lépésben az alábbi rész jelenik meg: \ref{fig:maintanceSupervisor} \\

\begin{figure}[h]
	\centering
	\includegraphics[width=11.8truecm, height=6.5truecm]{images/road6.png}
	\caption{Road Transportation Maintance Supervisor Approve}
	\label{fig:maintanceSupervisor}
\end{figure}



Itt a kiolvasott adatok mellett megjelenik egy újabb kitöltő mező, amelyben a Supervisor Descripont
kell megadni. \\
A következő 3. lépésben pedig azt kell megadni, hogy a probléma típusát, amely lehet Személyes,
Belső, vagy külső probléma.  \ref{fig:coordinatorApprove} \\
Amennyiben a probléma típusa Belső, vagy Külső, akkor egy újabb mező ugrik fel, amelyben szintén a
meglévő alkalmazottak közül kell kiválasztani azt, hogy a folyamat kihez megy tovább, ki lesz érte a
felelős. \\
A megjelenő oldalt itt láthatjuk :  \\
\newpage
\begin{figure}[h]
	\centering
	\includegraphics[width=14.8truecm, height=6.5truecm]{images/road7.png}
	\caption{Road Transportation Maintance Coordinator Approve}
	\label{fig:coordinatorApprove}
\end{figure}



\section{KÉP}

Tesztelésként kiválasztjuk az Internált, majd továbbítjuk, akkor az általunk megadott External \ref{fig:mechanicalApprove}, fog a
folyamat tovább menni.  \\

\begin{figure}[h]
	\centering
	\includegraphics[width=14.8truecm, height=6.5truecm]{images/road8.png}
	\caption{Road Transportation Maintance Coordinator Approve}
	\label{fig:mechanicalApprove}
\end{figure}



Ezután, az illetékes az alábbi formot fogja látni : \\
{\large
	\begin{center}
		\vglue 1truecm
		\textbf{\huge\textsc{Road Transportation Maintance utolsó lépése}}\\
		\vglue 1truecm
		\includegraphics[width=15truecm, height=7truecm]{images/road9.png}\\
		\textbf{\textsc{Approve page}}
\end{center}}


Itt az alábbi adatokat megadva, visszatér a folyamat az előző lépésben lévő emberhez, aki megkapja
ezeket az adatokat, és jóváhagyja. \\
Ezáltal vége a folyamatnak. Amennyiben a „Personal” részt választotta volna ki, akkor ott lett volna
vége a folyamatnak. \\
