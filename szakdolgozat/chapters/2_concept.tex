\Chapter{Folyamatok matematikai leírása}
A fejezet bemutatja, hogy a folyamatok modellezésére milyen
matematikai jelölések és modellek állnak rendelkezésre. Ebben a részben leginkább a strukturális programgráf, és a folyamatokhoz megfeleltetett automatákról, és azok megtervezéséről lesz szó lesz részletesebben szó.
\Section{Véges Determinisztikus automaták}

Véges számú állapotból, és a közöttük lévő átmenetekből (tranzíciókból) áll, amelynek szakaszait úgynevezett input szimbólumok jellemzik, amelyek ugyanabból a véges $\Sigma$ ábécéből vannak kiválasztva. 

Két kitüntetett állapot van: A kezdő állapot, jele : $q_{0}$, és a végállapot halmaza, amelynek a jele $F$. (Ezek a \textit{Start} és a \textit{Stop} elemnek tekinthetők egy folyamat modellezése esetén.
A \textit{Start} jelzi a folyamatábrán a \textit{draft} lépést, ahonnan elindul maga az üzleti folyamat. A végső lépés, az utolsó jóváhagyó, \textit{approve} lépés pedig a végállapot, amely megfeltethető az $F$-nek.)

Minden állapotban minden egyes bemeneti szimbólumhoz pontosan egy következő állapot tartozik. Diagram esetében pontosan ezek az állapotok közötti nyilak.

\Section{Formális definíció}

Egy véges determinisztikus autómatát a következőképpen definiálunk.
Legyen $A = (Q, \Sigma, \rho, q_{0}, F)$, ahol $Q$ az állapotok halmaza, $\Sigma$ a véges ábécé, $\Sigma \notin \emptyset$, továbbá $\rho$ a tranzíciós függvény, a  $q_{0}$ a kezdő állapot, $q_{0} \in Q $  $F$ a végállapot. $F \notin \emptyset$.

Könnyen belátható, hogy $F \in Q$.

A matematikai modell pontosan megfeleltethető annak, hogy egy folyamat esetén melyik lépésben milyen input adatokat ad meg a felhasználó, mert a következő lépés ezen adatok alapján változik. Például, egy konkrét esetben, ha a felhasználó 300'000 Ft-ot állít be egy becsült költségnek, akkor oda egy igazgatói jóváhagyás szükséges, ellenkező eseben ez a lépés kihagyható. Az ilyen elágazásokat, és ciklusokat követően jutunk el a (modellben $F$-el jelölt) végállapotba.

\Section{Egyszerű átmeneti rendszerek}

A folyamatok egy másik reprezentálási módját adják az átmeneti rendszerek.
Legyen $A = (S, T, \alpha, \beta)$, ahol
\begin{itemize}
	\item $S$: az állapotok véges, vagy végtelen halmaza,
	\item $T$: az átmentek véges, vagy végtelen halmaza,
	\item $\alpha, \beta$: $T \rightarrow S$ leképzés $\forall t \in T$-re, amelyben $\alpha(t)$ az átmenet kezdőpontja, $\beta(t)$ pedig a végpontja.
\end{itemize}

A modell ebben az esetben is egy irányított gráf, ahol a csúcsok állapotokat reprezentálnak, az élek pedig átmeneteket.

\begin{figure}[h!]
	\centering
	\includegraphics[width=8.5truecm, height=8.5truecm]{images/AllapotAtmenet.png}
	\caption{A 3.lépésben járó üzleti folyamat}
	\label{fig:3lepesben}
\end{figure}

Látható, hogy az elvégzett lépéseket zölddel, az aktuális lépést pedig pirossal jelzi a rendszer. \ref{fig:3lepesben} \\
$S:{\{ \text{ Draft,Supervisor Approve, Maintance Coordinator Approve, Mechanical Maintance, \\ Maintance Planning, Maintance Coordinator 2 } \}}$


% TODO: Megnézni, hogy a következő kommentezett rész milyen formában kerül további felhasználásra! (Csak akkor kell bele, hogy ha később lesz valamilyen jelentősége.) (Erre nincs szükség, mivel nem térek ki sehol utak konkatenációjára.)
% Az $A=(S,T,\alpha,\beta)$ átmeneti rendszerben $n>0$ hosszú útnak nevezünk egy olyan $t_{1},t_{2},\dots,t_{n}$ sorozatot, ha $\beta(t_{i}) = \alpha(t_{i+1})$ teljesül $\forall i\in \{ 1,\dots ,  n-1 \}$-re. \\
% Utak konkatenációja, vagy összefűzése $c * c' = t_{1}t_{2}\dots t_{n} g_{1}\dots $ \\

Jó példa egy egyszerű átmeneti rendszer egy italautómata, amelybe egy rögzített összeget dobva két gombot nyomhat meg a felhasználó.

% TODO: Nem teljesen látszik az FSA-val való kapcsolat.


\Section{Üzleti folyamatokhoz tartozó folyamatábrák}
A két modellezett üzleti folyamatokhoz tartozó folyamatábra pontos lépéseit az alábbi két ábra jelöli \ref{fig:Demand_Process} \ref{fig:maintane_process}. Itt pontosan látszik, hogy milyen jóváhagyási lépések lesznek szükségesek.
\begin{figure}[h!]
\centering
\includegraphics[width=7.5truecm, height=7.5truecm]{images/Demand_Process.png}
\caption{Road Transportation Demand}
\label{fig:Demand_Process}
\end{figure}

\begin{figure}[h!]
\centering
\includegraphics[width=7.5truecm, height=7.5truecm]{images/Maintance_Process.png}
\caption{Road Transportation Maintance}
\label{fig:maintane_process}
\end{figure}





\newpage
\Section{Folyamat megtervezése}

A folyamat tervezésénél fontos szerepe van a strukturáltságnak, ellenkező esetben nagy problémák is adódhatnak, mint egy  végtelen ciklus, esetleg több input bemenet.
\begin{itemize}
	\item Ha több input bemenet lenne (több különböző draft page), egy folyamaton belül,  akkor az már egy más folyamathoz tartoznak, egy más igényléshez.
\end{itemize}

A nagyobb és komplexebb folyamatokat, kisebb részekre is bonthatjuk, amelyre csak struktúrált programként van lehetőségünk megadni. Ezt az eljárást
a vezérlőgráf lebontásának nevezzük.

% TODO: A következő rész akkor lehet érdekes, hogy ha egy gráf validátort össze sikerül hozzá rakni.



Megjegyzés:
minden nem strukturált program átírható vele ekvivals strukturált program formályában. (Böhm-Jacopini tétele) \\
Áttérésként a Véges Determinisztikus autómatákra, egy $a\in \Sigma$ betűhöz, és $q_{i}$ állapothoz, pontosan egy output (nyíl) tarotzik. \\
% TODO: Ennél a Böhm-Jacopini tételt lehet hivatkozni. (Így emlékszem, a jegyzetet nem találom)
Minden valódi program átalakítható vele ekvivalens strukturált program formájába.

A Road Transportation Maintance folyamathoz tartozó jelölés a 3 alapvető elem segítségével :


\Section{Nem determinisztikus autómata}

Nondeterminisztic Finite Automation (NFA) esetén, viszon egy adott $q_{i}$ állapothoz és egy $a\in \Sigma$ jelhez több output is tartozhat, vagy egy sem.
\vskip 0.3 true cm
Tehát lehetséges olyan belső állapota, amiből a beolvasott szimbólum hatására több lehetséges állapot egyikébe mehet át.\\
Láthatjuk , hogy a véges állapotú autómaták megfeleltethetőek a modellezett folyamatoknak.
Egy strukturálatlan alapelemekből felépülő folyamat, megfelel egy nemdeterminisztikus automatának.
A strukturált elemekből felépülő modellezés pedig egy véges állapotú, determinisztikus automatának.

Mint ahogy a Struktúrált programozánál is említettük, hogy bármely nem struktúrált program átírható, vele ekvivalens struktúrált programmá, ez szintén jellemző a véges állapotú, nondetermenisztikus autómatákra.
Ezt a tranzíciós függvény elkészítésével tudjuk megoldani, amelyben minden olyan állapot, ahonnan egy bizonyos beolvasott szóval több helyre tudunk eljutni, egy új állapotként felvesszük, majd rekurzívan tovább vizsgáljuk.



\Section{DFA és NFA ekvivalenciája}

Legyen adott egy $N = (Q, \Sigma, \rho, q_{0}, F) $ véges nondeterminisztikus autómata. Nyelve legyen $\alpha(n)$. $n$-ből kiindulva szerkezhetünk egy olyan $A$ véges determinisztikus autómatát, amely ugyanazt az $\alpha(n)$ nyelvet fogadja el.\\
\begin{equation}
	A=(P(q),\Sigma,\rho^{'},[q_{0}],F' )
\end{equation}
Ahol $Q$ a részhalmazok összessége, $F'$ pedig a $Q$ mindazon részhalmazai, amelyek legalább egy $F$ beli állapotot tartalmaznak (végállapothoz vezetnek).

\begin{align*}
	&& \rho'  \text{ így értelmezhetünk, minden } a\in\Sigma \\
	&&  \rho'([q_{1},q_{2},\dots,q_{n}],a) = \rho(q_{1},a) \cup \rho(q_{2},a) \cup \dots \cup \rho(q_{k},a)
\end{align*}

Egy $A$ determinisztikus autómata és az N nem determinisztikus autómata által elfogadott nyelv ugyanaz, $\alpha(A) = \alpha(N)$.

\Section{Folyamatmodellezés}

A modell, a való világ egy részének egyszerűsített példánya.

% TODO: Hogy ha a modellről ilyen formában szó esik, akkor azt érdemes korábban megtenni. (Előrébb hoztam, a második részhez.)

A modellezés célja az, hogy felmérjük, majd  elemezhessük és javíthassuk a
folyamatainkat.Esetünkben azt célszerű vizsgálni, milyen jóváhagyási lépések szükségesek , és melyek azok amelyek elhagyhatóak, redundánsak. Az elmezés során több szempontot érdemes szemügyre venni. Például felhasználók kérését, az erőforrások kezelését.  A folyamatmodellek különböző információkat hordozhatnak és változó lehet a befogadó fél is, így nem mindegy, hogy milyen szemszögéből nézve készítjük el őket. Tehát a különböző nagy részlegeknek más-más célja lesz az adott folyamattal. Különböző megjelenési formákat kell biztosítani, más változókkal, mezőkkel , adatbázissal, és erőforrásokkal. A folyamatmodellek alapvetően két csoportra bonthatóak. Léteznek \textit{As-Is}   
modellek, amik a jelenlegi helyzetet mutatják be, és \textit{To-Be} modellek, amik a kívánt
szituáció ábrázolását takarják.  \\
A folyamatok leírására több lehetőség van. Legtöbbször ezek
kombinációját használják a felelős személyek, ahelyett, hogy egyetlen egyhez
ragaszkodnának. \cite{Corvin} Készülhet szöveges folyamatleírás vagy táblázatos leírás is, de a
legcélravezetőbb a grafikus, modell-orientált leírás, amely segítségével sokkal
egyértelműbben, és átláthatóbban ábrázolhatjuk az adott folyamatokat. A BPM rendszer grafikus ábrázolást használ, hiszen folyamatábrákat használ, a jó áttekinthetőség érdekében. \\

\SubSection{UML szabvány}
Több különböző szoftvert vehetünk igénybe a grafikus ábrázolás elkészítéséhez, amelyek megkönnyítik az elkészítést, de ebből adódóan egy hátrányt is
magában hordoz a módszer, mégpedig azt, hogy a többféle ábrázolási módból adódóan,
nehezen értelmezhető modellek jöhetnek létre. Ennek kiküszöbölésére azonban már
bevezetésre kerültek olyan modellező nyelvek, szoftverek, melyek egységesítik az
ábrázolás módját, ilyenek például az UML (Unified Modeling Language) szabvány. 
Az UML szabványos,  általános célú modellező nyelv, üzleti elemzők, rendszertervezők, szoftvermérnökök számára. \cite{xml}

Segítségével tervezni és dokumentálni lehet a szoftvereket, folyamatokat. Az UML-ben modellek és diagramok adhatók meg, különböző nézetekben.
Az UML két lehetséges lehetőséget definiál a használati esetek között, az egyik a beillesztés a másik pedig a kiterjesztés.
Az UML a jelöléseknek gazdag választékát kínálja, mely segítségével a szoftverfejlesztés összes fázisa modellezhető. \\

Az üzleti folyamatok modellezésével az üzleti stratégia és az informatikai
rendszerek között olyan kapcsolat hozható létre, ami nagyban hozzájárulhat üzleti
értékünk növeléséhez. \cite{Corvin}

\Section{Folyamatmodellezés részei}
A következő fejezetben részletesen említésre kerülnek, hogy egy adott üzleti folyamat megtervezésénél mire érdemes odafigyelni. 
% NOTE: Ez a rész kissé váratlanul jön, főleg annak tükrében, hogy utána ismét állapotgépes modell következik majd.

\subsection{Érzékenység vizsgálat}
A folyamat tervezésénél érdemes az alábbi kérdéseket megválaszolni.
Mi történik, ha rosszul becsljük meg az erőforrásainkat? \\
Mik számítanak lényeges paraméterek? \\
Milyen erőforrásokat bővítsünk? (emberi vagy gépi)? \\
Mi történik, ha a modellparaméterek nem pontosak? 

Az érzékenységvizsgálat során az optimális erőforrások paramétereinek a megvizsgálása, hogy azok megváltoztatása milyen hatással lenne az optimális megoldásra.\\
A szerkezet változatlan a marad.


	Kétféle érzékenységvizsgálatot különböztetünk meg.
\begin{itemize}
\item a jobboldali konstansokra az-az a kapacitásokra vontkozó érzékenység vizsgálat.
\item A $b_{i}$ paraméterek helyett a $b_{i} + \lambda$ szerepel a jobboldalon (kapacitás). \\ Az-Az $\lambda$ milyen értékek közé eshet?
\end{itemize}
Mennyivel tudjuk csökkenteni bizonyos erőforrásokat és mit érünk el vele?
Fontos  a célfüggvény együtthatók érzékenységvizsgálata. \\

	Egy általános képlet, az n.-ik erőforrás vizsgálatához.

\begin{equation}
	\vec{b}+ \lambda \vec{u}_{n} \geq 0
\end{equation}

Érdemes mindig azt növelni, ahol az árnyékár magasabb lehet, esetünkben azt, ahol a legjelentősebb erőforrás növekedést tudjuk elérni.

\subsection{Mennyiségi analízis}

Mennyi eset dolgozható fel egy órán belül? \\
Mennyi extra erőforrásra van szükség? \\
Mennyi az átlagos befejezési idő az egyes eseteknél?  \\
Alternatív megvalósítások esetén melyik modell képes azonos idő alatt több eset kezelésére rövidebb idő alatt feldolgozni egy esetet kisebb várakozási időket generálni ? \\


\subsection{Valószínűségi modell}

A folyamat során bizonyos lépéseknél, elemzéseket végezve, az erőforrásokat különféleképpen is feloszthatjuk, hiszen lehet, hogy egy elágazás során az egyik részbe az esetek $90 \% $-ban lép be, így erre a részre nagyobb erőforrást teszünk, mint a maradék $10 \% $ részre.
Az adatokat megvizsgálva következtethetünk, becsülhetünk.
Ez egy Diszkrét valószínűségi változó lesz, hiszen véges számú adatokat vizsgálunk.
