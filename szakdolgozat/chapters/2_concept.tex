\Chapter{Koncepció}

\Section{Folyamatok az iparban}

Minden nagyvállalatban egyszerre több meglévő folyamat zajlik. Ezeket a folyamatokat többféleképpen lehet nyomonkövetni.

Egy autógyárban lehet akár egy sima gyártási folyamat, akár egy karbantartási folyamat, vagy valamilyen igénylési folyamat, amelyek több lépésekben folynak le. Ezeket a folyamatokat valahol nyilván kell tartani, kezelni kell őket.

\Section{Az alkalmazás bemutatása}

Az fentebb említett problémára, egy olyan vállalati folyamatmodellezési rendszer készítése, amelyben végig lehet követni az elindított folyamatokat, lehetőség lesz új folyamatokat lehet elindítani, az elindított folyamatok állapotát megtekinteni , a meglévő folyamatokat nyomon követni, ezzel megkönnyítve a cég üzleti folyamatainak menedzselését.

A folyamatnak egy aktuális lépésében lehetőség van annak a továbbítása, amellyel a következő illetékes személyhez fog kerülni a feladat. (Ez akár egy feltételes elágazás is lehet, valamilyen megadott paraméter alapján).
Lehetőség van elutasításra, ha valamilyen hiba lépett fel, akár egy nem megfelelő adat, amellyel az előző személyhez kerül a feladat elküldése.

A folyamatot a vezetőknek lesz jogosultsága elindítani, és a vállalat alkalmazottai közül lehet majd választani, hogy kinek szeretné kiadni a feladatot. A folyamat elindításához szükséges kitölteni a feladat pontos leírását, az elvégzési határidőt, illetve kiválasztani az illetékes személyt.
 
Az alkalmazásban több különböző viewpage lesz megvalósítva.

Lesz egy belépő felület, amelybe az alkalmazottak felhasználónevét és a jelszavát kell megadni a belépéshez.  Különböző jogosultsági szintek lesznek megvalósítva.

A belépést követően a vezetőségnek, és az alkalmazottaknak is másféle weboldal fog megjelenni.

Felugró ablakban fog megjelenni a feladat elindítása.

Az alkalmazott a folyamatokat, amiket neki osztottak ki, megtekintheti egy erre specializált menüpontba, itt kiválasztva a folyamatot, megtekintheti a hozzá tartozó információkat.

\SubSection{Működése}

A felhasználó képes lesz arra, hogy a kapott feladatot továbbítsa, esetleg vissza küldje attól a felhasználótól (Vezetőségtől), akitől megkapta.

Egy feladat az alábbi információkat tartalmazza:
\begin{enumerate}
\item Feladat pontos leírása
\item Elvégzési határidő
\item Kitől kapta (Név + Részleg) 
\item Mikor kapta 
\end{enumerate}

Minden Részlegnek lesz egy vezetője, aki képes lesz ki listázni a részlegen lévő alkalmazottakat, itt megtudja tekinteni ki lenne a legaktuálisabb a feladat elvégzésére. Az egyes alkalmazottakra kattintva, lehetőség lesz az alkalmazott adatainak részleges megjelentésére. Ez a mód csak vezetőként belépve lesz elérhető. Itt tudunk majd egy feladatot elindítani. 

\SubSection{Megvalósítás}

Az adatok tárolása a Microsoft Sql Server-ben történik, itt egy CompanyDB-ben létre lesz hozva több különböző táblázat, mint például , Employee, Leader, Task, illetve Részleg.

A részleg táblázatban
\begin{itemize}
\item A részleghez tartozó egyedi azonosító : ID
\item A részleg pontos megnevezése, amelyhez az Id tartozik
\end{itemize}

A task táblához fog az alábbi rekordok fognak tartozni:
\begin{itemize}
\item Egy részleg vezetőéhez tartozó Id, hogy melyik vezető indította el a task-ot.
\item Egy időpont, hogy mikor lett elindítva a feladat.
\item Feladat pontos leírása, Description rész.
\item Illetve egy határidő rekort.
\end{itemize}

Employee Táblázatban:
\begin{itemize}
\item Id, Az alkalmazott egyedi azonosítója.
\item Az alkalmazott pontos neve. (Vezetéknév + Keresztnév) 
\item Életkora
\item Neme
\item Lakcíme
\item Telefonszáma
\item Email címe
\item poziciója
\item A részleg Id-ja, hogy melyik részlegen dolgozik az adott alkalmazott.
\item A részleg neve, amelyhez a megadott Id tartozik.
\end{itemize}

Leader Táblázatban:
\begin{itemize}
\item Leader pontos neve, ez tartalmazza a keretnevét, illetve a vezetéknevét. Összetett rekord lesz.
\item Részleg kódja, amely részlegen az adott személy vezető.
\item Elérhetősége , amelye egy email cím formát fog jelenteni, validálással megfeleltve.
\item Egy dátum mező, hogy pontosan mióta vezető az adott részlegen.
\end{itemize}

Egy Department tartalmazó táblázat:
\begin{itemize}
\item Id, A részleg egyedi azonosítószáma (Elsődleges kulcs)
\item Name, a részleg pontos megnevezése
\end{itemize}

Egy Task-ot tartalmazó táblázat:
\begin{itemize}
\item Egy Id, amely a task-ot azonosítja. Ki az elküldő, a feladatot ki kapja majd meg.
\item Description, a feladat pontos leírása.
\item Kezdő idő , amikor a feladat létre lett hozva.
\item Befejezési idő, hogy a feladat meddig végezhető el.
\end{itemize}

A kapcsolatok az alábbi séma szerint lettek kialakítva:

\begin{figure}[h]
\centering
\includegraphics[width=\textwidth]{images/er.png}
\caption{ER modell}
\label{fig:cimer}
\end{figure}

 Az adatbázisban a kapcsolatok kialakítása, és a feltételekhez kötött listázások .net framework egyik sajátosággával, az Entity Frameworkkel lesz kialakítva, Visual Studióban implementálva, asp.net Core-t használva.

\Section{Funkciók}

Az applikációban megvalósításra kerülnek az alábbiak:
\begin{enumerate}
\item Felhasználók kilistázása részlegek szerint, amelyre vezetőknek van jogosultsága.  Ez egy különálló view page-n fog megjelnni.
\item Lehetőségük lesz egy adott alkalmazott megkeresésére, majd az alkalmazattok adatainak részleges lekérésére.
\item Az olyan kiosztott feladatokat lehet kilistázni, amelyeket az aktuális vezető indított el . A feladatnak megtekinteni a feladatok státuszát,  mikor lett kiosztva , ki kapta meg a feladatot stb.
\item Alkalmazottak adatainak módosítása, esetleges alkalmazottak törlése, hozzáadása.
\item Feladat elindítása, itt egy felugró ablak lesz, amelyben a szükséges adatokat kell kitölteni a vezetőnek. (Validálással). 
\item A kiosztott feladatot az alkalmazott belépve megtekinthet, majd ha elvégezte a feladatot, akkor továbbíthatja a következő felhasználónak, visszaküldheti a vezetőnek. A feladat utolsó lépésben mindig a vezetőhöz kerül vissza az elvégzett folyamat.
\item A vezető végig tudja követni a feladat lefolyását.
\item Bejelentkező felület elkészítésé validálással, különböző jogosultságok megkülönböztetése.
\end{enumerate}

