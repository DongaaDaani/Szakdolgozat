\Chapter{A folyamatkezelő rendszer implementálása}
\section{Tervezés}

A projekt felépítése több komponensből tevődik össze.
Model rész, ahol az adatbázisnak megfelelő osztályok implementációja történik.
\\ A Controller rész, amelyben a szerver felé irányuló kérések implementációja kerül.
\\Pár alapvető kérés, mint a Post , amelyel egy új rekordot viszünk fel,  PUT amelyel egy meglévő adatot módosítunk , Delete amelyel törlünk , Get amely lekérdezi, vagy listázza az adatokat. Fontos a kivételkezelés ezekben a metódusokban, például, ha egy nem létező azonosítójú mezőt szeretnénk módosítani, akkor egy kivételt dobjon, és ne a rendszer fagyjon le, esetleg álljon le.
Data részben vannak az adatbázisban lévő kapcsolatok kialakítva, az itt definiált osztály, a DbSet osztályból származik.
Migration, amely a projekt sikeres Build-elése , a NuGet consolbe beírt  'Add-Migration Initial' + 'Update-Database' parancs után jön létre, így kialakulnak a kapcsolatok, létrejönnek a táblák. (A connectionStringben megadott Serveren, a megadott adatbázis névvel)

A jelszavak, és egyéb bizalmasan tárolt adatoknál a .net Framework Authentication Hash Algoritmust használt a titkosításhoz.      \\




Az üzleti folyamatok kezelésére BPM rendszert (Business Process Management
system) használnak. Kétféle típus létezik: \\
BPM rendszer: az üzleti folyamatok tervezésére, szimulálására és optimalizálására
szolgáló rendszer. Ezek elsősorban az információ elemzők, építészek stb. Eszközei.
Alig kapcsolódnak közvetlenül a vállalkozás működési szintjéhez. \\
BPMS: olyan rendszer, amely lehetővé teszi az üzleti folyamatok tervezését, de ezen
folyamatok végrehajtását is (más néven munkafolyamat). Ezeket a rendszereket az
egész társaságban lehet használni. Az építészet szintjétől az operatív szintig. \\


Maga az kezelő felület megjelenése teljesen dinamikus. A statikus tartalom itt azokat a
tartalmakat jelenti, amelyeket az oldalon nem változnak vagy csak ritka esetben változnak,
ilyen például folyamatok megnevezése, és a hozzájuk tartozó view pagen lévő mezők, illetve
a különböző kilistázási és módosító műveletek, vagy esetleg a céget bemutató szöveg, cég
logó. \\
Ezek az adatok közvetlen a React html (jsx) fájlaiban szerepelnek. A dinamikus tartalmak
pedig az adatbázisból töltődnek az oldal felületére, ilyenek például a felhasználók által felvitt
értékek egy folyamat során , akár az adatok betöltődésénél egy kiolvasási rész, akár egy
adatbázisba történő beszúrás, az első lépésben. A különböző dolgozó menedzselése, amely
során az adataikat lehet változtatni, esetleg törölni az adatbázisból, vagy egy új dolgozót
hozzáadni ahhoz. \\
Ami pedig a dinamikus megjelenést illeti, amelyben a react js elterjedt bootstrapje nyújt
segítséget, ebben az esetben azt jelenti, hogy a felhasználó bármilyen készülékről is
használja a vállalati alkalamzást (akár egy kisebb irodai laptop, akár egy nagy monitorral
rendelkező számítógép, Tablet ) , a tartalom átláthatóan és érthetően, a készülék méretéhez
igazítva jelenik meg. Nem lát kevesebb tartalmat és nem korlátozottabb az alkalmazás
használatában, egyszerűen csak egy más elrendezésű felületet kap, ami az adott készülékhez
igazodik. \\

A react js bootstrapjében rengeteg előre definiált stílusú objektumot tudtunk felhasználni,
amelyektől a weboldal sokkal felhasználó barátabb lesz a dolgozók számára. \\
Maga a felület a minimalista megjelenésre törekszik a könnyebb és átláthatóbb tartalom
közvetítése és használhatósága végett. \\
A vállalat dolgozóinak lehetősége van bejelentkezni a programba. Ezt email címük és jelszó
segítségével tudják megtenni. Ezután fogja tudni ténylegesen használni a programot.
Amennyiben hibás adatokkal próbál valaki belépni, egy hibaüzenet ugrik fel: "Érvénytelen
adatok. Adjon meg egy meglévő dolgozó adatait, vagy vegye fel a kapcsolatot az
ügyfélszolgálattal." \\

Bejelentkezés követően a felhasználó megtekintheti a kapott feladatokat egy külön erre
szolgáló menüpontban. Továbbá amennyiben jogosult rá, lehetősége van a rendszerbe
implementált folyamatok közül elindítani egyet. Két alapvető folyamat lesz definiálva a
rendszerben, egy személy gépjármű igénylő és egy karbantartási folyamat. \\
Több felhasználót különböztet meg a rendszer. \\
Erős felhasználó : Egy cégnél dolgozó felhasználó, akinek a szándéka a program tényleges
használata, a feladatok megtekintése, esetleg folyamatok elindítása. \\
Vezető : ő a programot egész átfogóan használja, annak minden egyes funkciójával együtt.
Lehetősége van a cégben dolgozó személyek adatainak módosítására, törlésére, különböző
lekérdezéskre. \\

A rendszer további előnye a felhasználóbarát szemléletnek köszönhetően különösebb
betanulási időt nem igényel, legfeljebb 5 perc alatt mindenki képes lesz elérni a kívánt
tartalmat, folyamatot elindítani, a feladatokat megtekinteni. \\
A folyamat elindításához szükség van az előre leimplementált draft űrlap minden kötelező
mezőjének a kitöltése, majd a submit gombra kattintva elküldheti a felhasználó a következő
illetékes személynek a folyamatot. \\
Ezután a program az adatokat lementi az adatbázisba, hogy a következő jóváhagyónál ezt
kiolvassa, és megjelenítse neki a felvitt adatokat, ha mentési hiba történne, akkor azt egy
hibajelző üzenettel a felhasználó tudomására adná a program. Hibaüzenettel alapvetően nem
találkozni, a dátum formátumra egy külön, React Bootstrapbe beépített rész lesz alkalmazva, a
helytelen formátum megadás elkerülésének érdekében. Továbbá az Érvénytelen számot adott
meg és hasonlók. \\

\section{React.js technológia részletesebb bemutatása} 
 A \textit{React.js} egy Javascript könyvtár, amelyet a \textit{Facebook}, \textit{Community} és egyéb hasonló
vállalatok tartanak fent. 
Az egyik különlegessége, hogy egy bizonyos \textit{„jsx”} típussal tér vissza, amely egy html kód. 
 \subsubsection{Futáshoz szükséges indítóparancsok} 

\begin{itemize}
	\item npm create-react-app
	\item npm install –g create-react-app
\end{itemize}
 
Egy react csomagot kell telepíteni, mivel ez is egy node package, ezzel a paranccsal
telepítjük. Ez a facebook fejlesztése, hogy mindenféle bonyolult konfiguráció nélkül tudjuk
elkezdeni a fejlesztést, mivel ezt már alapból megcsinálja. \\

\subsubsection{A következő lépésben létrehozzuk a projektet}

\begin{itemize}
	\item npx-create-react-app application-name
\end{itemize}
Tartalmazza a node modulokat, és a hierarchikus mappaszerkezetet.

A mappába belépve, az „npm start” parancsot kiadva tudjuk szerveren elindítani a projektet. \\
A node$\_$modules csomagokat a package.json mappában lévő 3 fő függőség

\begin{itemize}
	\item react
	\item react-dom
	\item react-scripts
\end{itemize}
 alapján telepíti fel. \\

A package.json-ben vannak korábban említett, már előre megírt scriptek.
\begin{itemize}
	\item Start, \textbf{npm start} - Ezzel a parancsal lehet a szerveren elindítani a projektet.
	\item Build,  \textbf{npm run build} - A végleges változatot fordítja le, ezzel egy olyan állapothoz lehet jutni.
	\item Test, \textbf{npm test} - A tesztet futtatja le.
	\item eject \textbf{npm run eject} - A konfigurációs fájlokat át lehet írni, de ezt már nem lehet visszavonni.
\end{itemize}




Mint a legtöbb Javascript keretrendszerben, a \textit{./src} mappában vannak js kiterjesztésű , fő fájlok. 
\textit{ES6} szintaktikát használ a \textit{React}, hiszen a különböző komponenseket importáljuk be, majd ez
automatikusan átfordul egy korábbi scriptre, hogy a böngészőkön könnyebben futhassanak. 
Több komponenst különböztetünk meg. 
Minden Komponensnek kötelező jelleggel van egy Render() függvénye, amelynek a
visszatérési értéke egy „html” , amely úgynevezett JSX. \\


Ennek az előnye az, hogy átláthatóbbá és könnyebbé teszi a kódolást. \\
\subsubsection{React dom kezelés szintaktikája} 

Az egyik legáltalánosabb metódusa a .render , amely 2 bemeneti paramétert vár. \\
Az első bemeneti paraméter, az maga a JSX kód, a második pedig az, hogy hova generálja az
adott kódot. \\

\begin{python}
ReactDom.render(JSX ,(document.getElementById('Id'))) 
\end{python}

Vannak bizonyos komponensek, ahol nem szükséges a Render methódus, ez az úgynevezett
Stateless Komponens, amely Önmagában egy függvény. \\
\begin{python}
const Vehicle =()=>{ 
	return ( 
	<div> Text </div> 
	) 
}
export default Vehicle; 
\end{python}
Ki kell adni (Exportálni), hogy ezt a komponenst további helyekre is belehessen importálni, a
következő paranccsal: \\

\begin{python}
import Vehicle from './path' 
\end{python}

Az importálást követően, ha hivatkozni szeretnénk a komponensen belül a Vehicle
függvényre, akkor azt a Render metóduson belül, html Tag-ként hivatkozhatunk , \textbf{<Vehicle/>}
szintaktikával tehetjük meg. \\
Amennyiben a stateless komponensnek tulajdonságokat szeretnénk átadni, akkor azt a
paraméter listában kell jelölnünk, ekkor az alábbira módosul a függvényünk. \\

\begin{cpp}
const Vehicle =(props)=>{ 
	return ( 
		<div>
			 {props.type}
		</div> 
		) 
	} 
export default Vehicle; 
\end{cpp}
A függvény kifogja írni a beérkező props-nak a type-ját. 
A render methódusban pedig a tagre az alábbi módon hivatkozhatunk : 
\textit{<Vehicle type=”opel” />} 


Amennyiben egy sima komponenst hozunk létre, több különböző beépített metódust tudunk
használni. (Ezek az ős komponensből származnak ) \\
Ilyenek például az alábbiak : 
\begin{itemize}
	\item \textbf{ComponentWillMounth} - A metódusban definiált tevékenységek lefutnak a Render függvény előtt.
	\item \textbf{ComponentDidMounth} - A metódusban definiált tevékenységek, a Render függvény után
	futnak le.
\end{itemize}
 
További előnyei, hogy a komponensen belül létre tudunk hozni construktort illetve state-et és
egyéb más dolgokat. \\
\section{Adatbázis használatának bemutatása}
Az implementáció során a Microsoft Sql Szerver adatbázis-kezelő szerverre jutott a döntés,
mivel az Entity Frameworkkel törénő adatbázis kezelés egy magas szintű absztakciót nyúlt,
programozási technika adatok konvertálására Visual Studioban, nem beszélve a kiváló
támogatottságáról, illetve a gyorsasága és megbízhatóságáról. Az ingyenes verzióban pedig
minden szükséges funkció elérhető a programunk működtetéséhez. \\
Amikor elkezdünk objektum orientált nyelvben programozni, akkor az adatkezelési feladatot
álltalában az objektumok módosításával végezzük.
Ez az úgynevezett Object-relational mapping (ORM).\\

\subsubsection{Entity Framework rövid bemutatása }

Létre kell hozni egy ConnectionStrings-et , amelyben a Servert nevét, és az Adatbázis nevét kell megadni, $+$ egy Truested Connection értéket. \\


Az adatbázisban lévő kapcsolatok az alábbi methodus felüldefiniálásával lettek megadva.
Létrehoztuk az adatbázisban lévő táblákat osztályonként, majd egy adott osztálybeli DbSet listát hozunk létre. \\
\begin{python}
public DbSet<Employee> Employee { get; set; }  
public DbSet<Department> Department { get; set; } 
public DbSet<Duty> Task { get; set; } 
public DbSet<Leader> Leader { get; set; } 
	
protected override void OnModelCreating(ModelBuilder modelBuilder) 
{
		
	base.OnModelCreating(modelBuilder); 
	modelBuilder.Entity<Employee>() 
	.HasOne(p => p.Deprartment) 
	.WithMany(b => b.Employee) 
	.HasForeignKey(ug => ug.DepartmentId) 
	.OnDelete(DeleteBehavior.NoAction); 
	...
}
\end{python}

Az osztály amelyben ezt a methódus megtalálható, egy DbContext osztályból származik le, amely implementálja az Entity Framework sajátosságait.
Fontos az 1:N kapcsolat kialakításánál, hogy az osztály adattagjaként felvegyük egy osztálytípusú listát, amelyből a több származik az egy felé.

\newpage

Az adatbázishoz tartozó rekordokat osztályként hozzuk létre.
Az Entity Framework sajátossága, hogy listaként kezeljük az adatbázist, így egy "Select * from DbName" parancs helyett elég meghívni a listának a .toList() methódusát.
Hasonló módon, ha egy rekordot szeretnél az adatbázisba felvinni, akkor a a lista.add() methodussal lehet megtenni.
További a .remove , .hasFirstDefault stb.. \\
Az Entity Frameworkhoz való hozzáférés a nyelvbe integrált (linq) segítségével történik.
\subsubsection{HTTP POST kérés szintaktikája} 

\begin{python}
[HttpPost] 
public async Task<ActionResult<Department>>
PostDepartment(Department department) 
{ 
	_context.Department.Add(department); 
	await _context.SaveChangesAsync(); 
	return CreatedAtAction("GetDepartment",
	new { id = department.DepartmentId }, department); 
}  

\end{python}

Látható, hogy a Department rekordot feltölti a kapott Department osztálytípusú department nevű objektummal a  
\begin{python}
	
_context.Department.Add(department); 
\end{python}
résznél.


Összehasonlítva a hagyományos módszerrel, az ORM gyakran lecsökkenti a megírandó kód
mennyiségét, objektumorientált nyelv és relációs adatbázis közötti adatcsere esetén.
Az ORM hátrányos az olyan területeken, rendszerekben ahol az adatbázist jelentősen
optimalizálják speciális technikák alapján.  \\

Ahhoz, hogy az Entity Frameworköt használatba tudjunk fenni, különböző szükséges \textit{NuGet} csomagokat kell telepíteni a programba, amelyeket a visual studióban megtalálunk, mint például: \\
\begin{python}
Microsoft.EntityFrameworkCore 
Microsoft.EntityFrameworkCore.Design 
Microsoft.EntityFrameworkCore.SqlServer 
Microsoft.EntityFrameworkCore.Design 
Microsoft.EntityFrameworkCore.Tools 
	
\end{python}

Az Entity Framework előnye, hogy az adatbázisban lévő adatokat lista szerkezetként tudjuk
kezelni, így a megszokott adatbázis parancsokat felcserélhetjük, egy egyszerűbb
szintaktikával rendelkező kódolásra. \\
\newpage
\subsubsection{Hagyományos SQL és az Entity Framework összehasonlítása} 
\subsubsection{Listázás}
\begin{python}
	Select * from TableName 	// SQL
	listName.toList()		//Entity Framework
\end{python}
Egy listázást az alábbi módon tudjuk megtenni, mind sql-ben , mind Entity Frameworköt használva.\\

\subsubsection{Beszúrás}
\begin{python}
	Insert Into TableName values	// SQL
	listName.Add()			//Entity Framework
\end{python}
Az adatbázisba történő beszúrás, egy  parancs helyett
könnyedén alkalmazhatunk egy  metódust, ahol a paraméter listába az adatbázisnak megfelelően egy osztály példánya kerül. \\


\subsubsection{Keresés}
\begin{python}
	 Select *from tablename where condition 	// SQL
	listName.where(condition)		//Entity Framework
\end{python}
Egy feltételekhez kötött listázás szintaktikája is tovább egyszerűsödik. \\

\subsubsection{Törlés}
\begin{python}
	DELETE FROM table_name WHERE condition     //SQL
	listname.delete(condition)		/Entity Framework
\end{python}

További sajátosságai, jelentősen megkönnyítik a szerver oldali implementációt, a kliens
oldallal történő kommunikáció használatára pedig json \\ (Javascript Object Notation)-t
használunk, amely tulajdonképpen rendezett érték lista amely , egy név és érték párok
gyüjteménye dupla idézőjel párok között megadva. \\

Általános szintaktikája : 
\begin{python}
{
	"name":"value", 
	"name2":"value2" 
	...
} 
\end{python}
Kezdetben a szerver oldali tesztelés, adatok felvitele, listázása és egyéb alapvető funkciók a
Postman nevű alkalmazásban történik. \\
Mivel a Json sokkal könyebben értelmezhető, mint az xml, ezért esett erre a választás. \\


További lényeges implementáció szükséges a szerver oldalon ahhoz, a frontend értelmezni
tudja a kapott értékeket. \\
Az alapvető Requesteket , a Microsoft által készített .NET keretrendszerben kerültek
implementálásra. \\
A keretrendszer előnyei közé tartozik szintén a megfelelő támogatottság, hiszen egy
Microsoft és a Xamarin áll mögötte, Mivel .NET Framework alapját a CLI, vagyis a Common
language Interface képezi, ezért a platformfüggetlenség, illetve a gyors alkalmazás
fejlesztés (RAD, Rapid application development). \\
\section{Platformfüggetlenség}
A CLI rendszerek leírják nyelvfüggetlen fejlesztői környezetet, a futtatókörnyezetet,
típusrendszert. \\
Három fő része van : \\
\begin{itemize}
	\item A \textit{CLS}, amely azokat a szabályokat írja le, amelyeket a \textit{CLI}-kompatibilis nyelveknek	be kell tartaniuk
	\item  A \textit{CTS}, amely a \textit{CLI} azon része, ahol a típusokat, és azoknak a memóriaban történő reprezentációját illetve egymással való kapcsolatait írja le.
	\item  \textit{CLR}, amely a \textit{CLI} nyelven megírt programok betöltéséért és végrehajtásáért felel. \cite{net}
\end{itemize}



A gyors alkalmazás fejlesztéshez használt módszertan 3 fő elemből tevődik össze: \\
Ciklikus fejlesztés, működő prototípusok létrehozása, és a szoftverfejlesztést támogató
számítógépes programok.\\

Előnye: Az alkalmazások elkészítése egyszerűbbé válik , amelyhez hozzátartozik egy
felhasználóbarát kezelőfelület, illetve egy uniformizált platform megjelenése. \\
