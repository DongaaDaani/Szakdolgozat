\chapter{Összefoglalás}
A szakdolgozatom implementációja során részletesebben megismertem napjaink népszerű technológiáit és keretrendszerét, a React-ot és a .NET keretrendszert.
A webalkalmazásban elkészített funkciók során, egy React Folyamatábrákkal kapcsolatos könyvtárat ismertem meg. Továbbá az autentikáció implementálását, és 
reszponzív űrlapok létrehozását is megtanultam. Szerver oldalon a táblázatok elkészítését, lekérdezését, és összekapcsolását, az entity framework használatával. Ezekkel az ismeretekkel a Full Stack fejlesztésben szerzett tapasztalatomat növeltem. Az alkalmazás elkészítése során a két modellezett üzleti folyamat űrlapjai elkészítésre kerültek, és további olyan hasznos funkciók, amelyek a vállalat irányítását megkönnyíthetik. Az alkalmazottaknak bejelentkezés, vagy regisztráció után lehetősége van használni az alkalmazást.  
 Tovább fejlesztés lehet az alkalmazás elkészítésében a különböző jogosultsági 
szintek megkülönböztetése, hogy elszeparálja a felhasználókat, a vezetőktől, ezáltal a feladat küldése is megvalósítható.
A szakdolgozat elméleti részében részletesen bemutatásra kerültek a különböző használt folyamatmodellek. Ezeket különböző matematikai módszerek segítségével
vizsgáltuk, annak érdekében, hogy jobban belelássunk az előnyeikbe és a hátrányaikba.
A folyamatmodellezés vizsgálatainál sok segítséget jelentett, a korábban már egyetemen megtanult valószinűségszámítási, és optimalizálási ismeretek.
Az elméleti rész során felhasználásra került, sok korábban tanult összefüggés, képlet és kritérium.  A szakdolgozatban részletesen taglalva van, hogyan lehet optimális megoldást találni egy adott probléma.
Továbbá az automatákkal kapcsolatos fejezetnél hasznos volt, az automaták és formális nyelvek kurzuson tanult definíciók, és összefüggések. 
Megismertük a Petri hálók működését, és használatát, továbbá a nem determinisztikus automatát, és a struktúráltatlan folyamatgráfot. 
A webalkalmazásban elkészített folyamatábrák segítségével tetszőleges üzleti folyamatokat tudunk elkészíteni, viszont ami ennek egy tovább fejlesztése lenne,
az a strukturáltsággal kapcsolatos vizsgálata lenne. 