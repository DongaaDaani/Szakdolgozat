\Chapter{User Interface}
\section{Design}
\section{Bevezetés a saját program vázának elkészítésébe}
Ebben a fejezetben bemutatásra kerül a folyamatmenedzsment rendszer
felhasználói felülete illetve az adatbázis modellje. A felhasználói felületet
képernyőfotókkal készítve, egy kisebb változatú, de hasonló célú projekttel
mutatom be, illetve mindegyikhez adunk egy-egy rövid leírást az adott felhasználói
felület funkcióiról. \\
A programunkhoz használt adatbázis leírása és bemutatása után pedig részletesen
bemutatjuk a programunk adatmodelljét, ER és relációs modell segítségével. \\

\section{2. Felhasználói felület}

Ebben a fejezetben bemutatásra kerülnek a felhasználói felület elemei, melyeken
keresztül a felhasználó a programot használni fogja. A felhasználói felület
kialakítása során törekedtünk az egyszerűségre és az átláthatóságra, hogy a
felhasználó számára a lehető legegyszerűbb legyen a program használata. Ez
csökkenti a felhasználó által használt programidőt, ezáltal programunk kevésbé lesz
hajlamos a lassúságra, illetve az esetleges összeomlásra. Ezért az egyes funkciókat
úgy próbáltuk megoldani, hogy azok a felhasználó ezen a területen szerzett előzetes
tapasztalatainak megfeleljenek. A React.js alapvető bootstrapjét használjuk a
projekt implementálásánál, amelyet a következő parancs telepítése után vehetünk
igénybe.

\begin{python}
npm install react-bootstrap bootstrap 
\end{python}
	
	majd beimportáljuk a projektbe. 

\begin{python}
	<link \\
	rel="stylesheet" 
	href="https://maxcdn.bootstrapcdn.com/bootstrap/4.5.0/
	css/bootstrap.min.css" integrity="sha384- 
	9aIt2nRpC12Uk9gS9baDl411NQApFmC26EwAOH8WgZ
	l5MYYxFfc+NcPb1dKGj7Sk" 
	crossorigin="anonymous" /> 
\end{python}	
	

Ezáltal képesek vagyunk responsive oldalt létrehozni, illetve felhasználni a
React.js material componenseit, amelyeket az alábbi oldalon találhatunk. \\
https://material-ui.com/components/snackbars/ \\

A részlegek felülete: \ref{fig:reszlegek} \\



\begin{figure}[h!]
	\centering
	\includegraphics[width=15truecm, height=6.5truecm]{images/ListDepartment.png}\\
	\caption{Részlegek listázása}
	\label{fig:reszlegek}
\end{figure}






Vezetőként belépve, lehetőségünk van a részlegek megtekintésére, törlésére,
módosítására. \\
Edit gomra kattintva, egy felugró ablak jelenik meg, amely kiolvassa, az aktuális
mezőhöz tartozó adatokat, ezek alapján lehet módosítani. \ref{fig:editreszleg} \\
A felugró ablak hasonlóképpen néz ki: (A mezőhöz tartozó DepartmentID
Disabled, hiszen azt nem lehet módosítani) \\

\begin{figure}[h!]
	\centering
	\includegraphics[width=11.8truecm, height=7.5truecm]{images/EditDepartment.png}\\
	\caption{Részleg szerkesztése}
	\label{fig:editreszleg}
\end{figure}


A törlés gomb megnyomásával egy Alert rész ugrik fel, rákérdez arra, hogy
„Biztosan szeretnéd törölni”? \\
Ugyanezen a formon lehetőségünk van egy részleg hozzáadására, az Id
automatikusan incrementálódik, tehát a felhasználónak csak magát a részleg
nevét kell megadnia. \\

Az Alkamazottak felülete \ref{fig:alkalmazottak} : \\



\begin{figure}[h!]
	\centering
	\includegraphics[width=15.5truecm, height=6.5truecm]{images/ListEmployee.png}\\
	\caption{Alkalmazottak listája}
	\label{fig:alkalmazottak}
\end{figure}



A design és a funkciók hasonlóak az előző fórumom láttottakhoz, a tárolt
adatok különböznek, illetve van még némi különbség. \\
Ha egy új dolgozót szeretnénk felvenni a rendszerbe, akkor a részleg
kiválasztásnál, a meglévő részlegek közül tudunk választani, egy legördülő
listában. \ref{fig:addEmployee} (Ezzel korlátozva azt, hogy hibás részleget adjon meg a felhasználó,
vagy nemlétezőt. ) \\


\begin{figure}[h!]
	\centering
\includegraphics[width=14truecm, height=7.5truecm]{images/addemployee.png}\\
	\caption{Alkalmazott hozzáadásának a felülete}
	\label{fig:addEmployee}
\end{figure}



Az adatbázisban felvett Dátum típus miatt problémák lehetnek azzal, ha
hibásan kezeljük a forumon leimplementált Dátum mezőt. \\
Ebből kifolyólag, egy Dátum típusú Form Control lett megadva, a helytelen
konvenciók miatt. \\
Ez az alábbi képen tekinthető meg :  \ref{fig:editEmployeees} \\


\begin{figure}[h!]
	\centering
	\includegraphics[width=10.8truecm, height=8.5truecm]{images/EditEmployee.png}\\
	\caption{Alkalmazott szerkesztésének a felülete}
	\label{fig:editEmployeees}
\end{figure}



Mivel két különböző szerveren fut a Backend és a Frontend, így a
kommunikációt Json formátum segítségével lesz megoldva \ref{fig:backendget}, amelyben a
Backend oldalon az alábbi rész kerül implementációra. \\
Ahhoz hogy a két server kommunikálni tudjon egymással, és ne írjon hibát, fel
kell oldani egy Cors problémát a .net résznél, hogy ez megvalósuljon. \\
\newpage
Mivel a frontend a 3000-as porton érhető el így az implementációja az alábbi
formátumban lesz, a nuget csomagok telepítése után :  \\
\begin{python}
	config.EnableCors(new
	 EnableCorsAttribute("http://localhost:3000", "*", "*")); 
\end{python}

A json formátummá alakításhoz pedig : 
\begin{python}
	config.Formatters.JsonFormatter.SupportedMediaTypes.Add( 
	new MediaTypeHeaderValue("text/html")); 
\end{python}


Így ha a fentebbi listát tekintjük meg Backend oldalon , akkor az alábbi Json \\
formátumot fogjuk látni :  \\

A department részleg esetén \ref{fig:backendgetDepartment} : 

\begin{figure}[h!]
	\centering
	\includegraphics[width=\textwidth]{images/DepartmentJson.png}\\
	\caption{Backend szerveren futó Get lekérdezés}
	\label{fig:backendgetDepartment}
\end{figure}


Az employee részleg esetén \ref{fig:backendget} : 

\begin{figure}[h!]
	\centering
	\includegraphics[width=\textwidth]{images/EmployeeJson.png}\\
	\caption{Backend szerveren futó GET lekérdezés}
	\label{fig:backendget}
\end{figure}


A Cliens oldali részről, ahhoz hogy ezt a Json formátumot használhatóvá tudjuk \ref{fig:fetchapi}
tenni, a \textit{ComponentDidMount} függvényen belül definiálni kell a \\
\begin{python}
fetch('https://localhost:44315/api/Department')
fetch('https://localhost:44315/api/Employee')
\end{python}

 A modellezett üzleti folyamatot a felhasználó kétféleképpen tudja ábrázolni az alkalmazásban. \ref{fig:fetchapi}



\begin{figure}[h!]
	\centering
	\includegraphics[width=\textwidth]{images/businessprocessDraw.png}\\
	\caption{Road Transportation Maintance ábrázolása az alkalmazásban.}
	\label{fig:fetchapi}
\end{figure}
A következő ábrázolásban a lépések közötti sorrend, és az elágazások jobban átláthatóak.

\begin{figure}[h!]
	\centering
	\includegraphics[width=8truecm, height=12truecm]{images/2nd process diagram.png}\\
	\caption{Folyamatábra készítése a Road Transportation Maintance-hoz.}
	\label{fig:roadtransporationdemand2ndabra}
\end{figure}




Amíg a Kliens oldali rész nem került implementálásba, addig az alapvető CRUD
metódusok tesztelése a Postman segítségével történt \ref{fig:postman} , ahol kiválaszthatjuk hogy
Post / Delete / Update / Delete methodust szeretnék végrehajtani , az adott
URL-en.

Itt egy adat kerül hozzáadásra, az Alkalmazottak mezőhöz. A post methodusnál
a kivételkezelés kifejezetten fontos, így hibás adat esetén nem kaptunk volna
sikeres visszajelzést.

	


\begin{figure}[h!]
	\centering
	\includegraphics[width=15.5truecm, height=4.5truecm]{images/PostmanJson.png}\\
	\caption{Postmanban történő tesztelés}
	\label{fig:postman}
\end{figure}





Az új igénylő első lépése megtekinthető a \ref{fig:newMaintance} , az első jóváhagyó lépése pedig a \ref{fig:newMaintance2} képeken tekinthető meg.
\newpage
\begin{figure}[h!]
	\centering
		\includegraphics[width=15truecm, height=7truecm]{images/newRoad1.png}\\
	\caption{Az új környezetben lévő Road Transportation Maintance}
	\label{fig:newMaintance}
\end{figure}

\begin{figure}[h!]
	\centering
		\includegraphics[width=15truecm, height=7truecm]{images/newRoad2.png}\\
	\caption{Az új környezetben lévő Maintance Approve lépés}
	\label{fig:newMaintance2}
\end{figure}

\newpage
\begin{python}
	componentDidMount(){
		this.refreshList();
	}
	
	refreshList(){
		fetch('https://localhost:44315/api/Draft')
		.then(response=> response.json())
		.then(data=>{ 
			this.setState({draft:data});  }); }
\end{python}


Telefon,vagy egyéb kisebb eszközön a design \ref{fig:responsive} 

\begin{figure}[h!]
	\centering
	\includegraphics[width=7.5truecm, height=14truecm]{images/Responsive.png}\\
	\caption{A responsive megjelenése az igénylő forumnak}
	\label{fig:responsive}
\end{figure}


