\Chapter{Petri hálók}

\Section{Bevezetés}
A Petri-hálókat az 1960-as években Carl Adam Petri határozta meg először. \\
A Petri háló (Petri net, PN) egy súlyozott, irányított élű páros gráf, melynek élei kétféle
csomópontot köthetnek össze: a körrel jelölt helyeket (places), illetve a téglalap-alakú állapotátmeneteket (transitions). \\
Formálisan a Petri hálók olyan <P,T,E,  $\delta $ , $m_{0}$> ötös, ahol
\begin{align*}
	P & \text{ a } p_{i} \text{ helyek véges halmaza } \\
	T & \text{ a } t_{i} \text{ tranzíciók véges halmaza } P \cap T = \emptyset \\
	E & : (P \times T ) \cup (T \times P) \text{ a P-t és a T-t összekötő élek halmaza.}\\
	\delta & \text{ a súlyfüggvény , }\qquad \delta:E\rightarrow N^{+} \\
	& \text{Token eloszlás  }  m:P\rightarrow N   \text{ ami egy olyan függvény, } \\ & \text{  amely adott állapotban megadja a helyeken lévő tokenek számát} \\
	&
\end{align*}
$m_{0}$ a kezdeti token eloszlás, az-az a hálózat kezdeti állapota. \\
A tranzíciót egy rövid vízszintes vonallal reprezentált csúcs.
\begin{align*}
	(p,t) \in & (P\times T) : \text{ p-ből t-be irányított él.} \\
	(t,p) \in & (T \times P) : \text{t-ből p-be irányított él.} \\
	m: P \rightarrow N & \text{ token eloszlás } m(p) \\ & \text{-t a p helyen } m(p)  \text{ darab korong reprezentálja az m állapotban.} \\
	e \in E & \text{ él címkéje a } w(e)
\end{align*}

\Section{Élcsoportok}

Két élcsoportot definiálhatunk a Petri hálóba, a bemenő és a kimenző élek halmazát. Ez megfogalmazva az alábbiakat jelenti : \\

\begin{align*}
	\text{Pre : } T \rightarrow \mu (P) & \text{ , ahol Pre(t) } = \{ p\in P \mid (p,t) \in E \} \text{ t őseinek vagy bemenő helyeinek halmaza}\\
	\text{Post: } T \rightarrow \mu(P) & \text{ ,ahol Post(t) } = \{ p\in P \mid (t,p) \in E \} \text{ t útódainak vagy kimenő helyeinek halmaza} \\
	\forall p \in P , \forall t \in T & \text{ esetén } \delta^{-}((p,t)) = \delta((p,t)) \text{ ha } (p,t) \in E \text{ egyébként } \delta^{-}((p,t)) = 0 \\
	& \delta^{-}((t,p)) = \delta((t,p)) \text{ ha } (t,p) \in E \text{ egyébként } \delta^{-}((t,p)) = 0
\end{align*}
$\Delta $ súlyozott szomszédsági $ \mid T \mid \times \mid P \mid $ dimenziós mátrixok, ahol \\ 
$\Delta(t,p) = \delta^{+}((p,t))-\delta^{-}((t,p)) $. \\
A $\Delta^{-} $ és a $\Delta^{+}$ szintén $\mid T \mid \times \mid P \mid $ dimenziós mátrixok, ahol $\Delta^{-}(t,p) = \delta^{-}((p,t))$ illetve $ \Delta^{+}(t,p) = \delta^{+}((p,t)) $  \\

Egy tranzíció tüzelés akkor megengetedd, a t minden ősén van legalább $\delta((p,t))$ token, vagyis
\begin{equation*}
	\forall p \in Pre(t) : m(p) \geq \delta((p,t))
\end{equation*}
m állapotban véletlen módon választ egy engedélyezett tranzíciót, melyet tüzel. \\
m állapotban a t traníció tüzelésének eredménye az-az m' állapot , ahol 
\begin{equation*}
	\forall p \in P-re: m'(p) = m(p) - \delta^{-}((p,t)) + \delta^{+}((p,t))
\end{equation*}




A Petri hálók állapotváltozók státuszát reprezentálják. Az állapotokat a hely körében lévő fekete
pontok, az úgynevezett tokenek reprezentálják. A helyállapota a benne lévő tokenek számát
jelenti. A hálózat állapota az egyes helyállapotok összessége. Az állapotvektor a  $ \tau = \mid P \mid $ komponensű M token-eloszlású vektor, ahol a $p_{i}$ helyen található tokenek számát jelöli $m_{i}$ \\

A Petri hálók működése állapotátmenetekkel (trajektória) reprezentálható. Egy állapot
megváltozása a tranzíciók tüzelését jelenti:

A petri hálókat szokták alkalmazni az alábbi rendszerek modellezésében :
1.Nemdeterminisztikus \\
2.Párhuzamos \\
3.Elosztott\\
4.Konkurens\\
5.Asszinkron \\
A Petri-háló diszkrét elosztott rendszerek matematikai ábrázolása.\\
Az áttekinthetőség kedvéért ad egy grafikus reprezentációt, a precizitás és egyértelműség kedvéért pedig egy matematikai reprezentációt. \\
Az ábrázolás az egy időben lezajló események megjelenítésére alkalmas, az automataelmélet általánosításának tekinthető. \\
Előnye az autómatákhoz képest, hogy az ábrázolás az egy időben lezajló események megjelenítésére alkalmas, az automataelmélet általánosításának tekinthető. Előnye az autómatával szemben, hogy sokkal szemléletesebben fejezi ki a szinkronizációt, illetve kompakt módon fejezi ki az állapotot.
\vskip 0.3 true cm
\Section{Működése és felépítése}
irányított,súlyozott, páros gráf\\
Két típusú csomópontja lehet, egy kör és egy téglalap. A kör esetén egy hely, téglalap esetén egy trazíció.\\
Van egy állapotjelölő token, amely fekete pötty a hely körébe rajzolva.
A hely állapota pedig a bennelévő tokenek száma.\
Petri hálók jellemzői : \\
Azonnal tüzelések , Aszinkron tüzelések , nem determinisztikus , két tranzíció nem tüzel egyszerre, nem interpretált.\\
Bármely $e\in E$ élhez egy $ \delta : E  \rightarrow \Re^{+}  $ súlyfüggvényt is hozzárendelhetünk. \\
A minimális súlyú feszítőfa megtalálására legismertebb algoritmus a mohó algoritmus, amelyet az alábbi képpen definiálhatunk :\\
Egy $\Gamma = (V,E)$ összefüggő gráf. Legyen $\delta R \rightarrow \Re^{+}$ egy súlyfüggvény. A mohó algoritmus lépésében, kiválasztunk egy olyan $e_{1} \in E$ elemet, amelyre $\delta(e_{1}) \leq \delta(e)  $ ,  $\forall e \in E $ \\ Ha megvan kiválasztottuk az $e_{1} , e_{2}, \dots , e_{k}$  éleket, akkor az $e_{k+1}$-edik élet úgy választjuk, hogy ne alkosson kört
\begin{equation}
	e_{k+1} = { e\in E \setminus \{ {e_{1},e_{2},\dots,e_{k}} \} | \text{ az } \{ {e_{1},e_{2},\dots,e_{k},e} \} \text{ élek nem alkotnak kört} }
\end{equation}
illetve teljesül az is , hogy $ e_{k+1} < \forall \delta(e)$ \\

Könnyen belátható ez az algoritmus n-1 lépésből áll, ahol \\  $n=|V|, \text{ amely egy } \{ e_{1},e_{2},\dots,e_{n-1} \} $  élsorozatú feszítőfát eredményez.
\vskip 0.3 true cm
Egy
\begin{align*}
	\Gamma = (V,E)  \text{ gráf feszítőfájának nevezünk egy olyan} \Delta \text{ részgráfot amelyre } \\ \Delta = (V,E) \text{ és } \Delta \subset \Gamma \text{ és } \Delta \text{ fa}
\end{align*}
Az állapot megváltozása a tranzíciók "tüzelésével" történik.
Tüzelés végrehajtása az alábbiképpen zajlik :
Első lépésben megvizsgálja az engedélyezettséget, majd ezt követően a token elvétele következik a bemetni helyről, majd a token kirakása a kimeneti helyekre.\\
Ez egy olyan új állapotot eredményez, amelyben megváltozik a Token eloszlás vektor.
\\ Ezzel a témával kapcsolatosan hasznos információkat lehet találni az alábbi linken : \\
https://inf.mit.bme.hu/sites/default/files/materials/category/kateg%C3%B3ria/oktat%C3%A1s/msc-t%C3%A1rgyak/form%C3%A1lis-m%C3%B3dszerek/11/PN_alapfogalmak_kiterjesztesek.pdf

