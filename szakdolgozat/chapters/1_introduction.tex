\Chapter{Bevezetés}

Minden vállalkozásban jelen vannak öszetett, és kevésbé öszetett üzleti folyamatok, amelyenek a bonyolultsága akár egy számlázástól egy teljes átszervezésig terjedhet.
Ezeket a folyamatokat a legtöbb cégnek valamilyen formában struktúráltan vezetni kell, vagy könnyen irányítását veszti, ezzel értékes erőforrásokat veszíthet a cég.
  
A szakdologozat célja az, hogy egy olyan információs rendszer készüljön el, amely segítségével nyomon követhetők, elemezhetők és így optimalizálhatók vállalatok üzleti folyamatai. Az alkalmazásban a vállalat alapinformációinak lekérdezésén felül, különböző szerepköröket, jogosultsági szinteket különböztet meg (amellyel például az alkalmazottak és vezetők hozzáférési módja, és elvégezhető műveleteik szabályozhatók).

A szakdolgozatban különböző matematikai, és optimalizálási megközelítéseket fogunk tárgyalni, amelyeket folyamatgráfokkal, és egyéb ábrákkal fogunk személtetni.
A modellezett folyamatok véges állapotú automatáknak tekinthetők. A dolgozat részletezi ezek megvalósításának gyakori módjait, az azokból adódó előnyöket és hátrányokat. A vizsgálatok kitérnek arra, hogy hogyan lehet kezelni a folyamatokban bekövetkező változásokat, automatikusan jelezni a hibákat, javaslatokat tenni a folyamatok hatékonyabbá tételére, ilyen módon optimalizálva azokat.

Az alkalmazás szerver oldali része C# programozási nyelven .Net környezetben készül. Az adatok perzisztens tárolásához Microsoft SQL szerver kerül felhasználásra, melyhez a hozzáférést az Entity Framework segíti. A kliens megvalósításához a React nevű JavaScript keretrendszert használja az alkalmazás.

% A fejezet célja, hogy a feladatkiírásnál kicsit részletesebben bemutassa, hogy miről fog szólni a dolgozat.
% Érdemes azt részletezni benne, hogy milyen aktuális, érdekes és nehéz probléma megoldására vállalkozik a dolgozat.

% Ez egy egy-két oldalas leírás.
% Nem kellenek bele külön szakaszok (section-ök).
% Az irodalmi háttérbe, a probléma részleteibe csak a következő fejezetben kell belemenni.
% Itt az olvasó kedvét kell meghozni a dolgozat többi részéhez.
