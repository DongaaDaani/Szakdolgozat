\Chapter{Bevezetés}

Minden vállalkozásban jelen vannak összetett, és kevésbé összetett üzleti folyamatok, amelyeknek a bonyolultsága egy számlázástól akár egy teljes átszervezésig terjedhet.
Ezeket a folyamatokat a legtöbb cégnek valamilyen struktúrált formában  vezetni kell, vagy könnyen elveszítheti az irányítást, mely értékes erőforrások elvesztését jelentheti a cég számára.
  
A szakdologozat célja az, hogy bemutassa egy olyan információs rendszer elkészítését, amely segítségével nyomon követhetők, elemezhetők és így optimalizálhatók vállalatok üzleti folyamatai. Az alkalmazásban a vállalat alapinformációinak lekérdezésén felül, különböző szerepköröket, jogosultsági szinteket különböztet meg (amellyel például az alkalmazottak és vezetők hozzáférési módja, és elvégezhető műveleteik szabályozhatók).

A szakdolgozat különböző matematikai, azon belül is elsősorban optimalizálási módszereket tárgyal, amelyeket többségében folyamatgráfokkal szemléltethetünk.

A szakdolgozatban elméleti részében a folyamatmodellezés pontos lépései lesznek tárgyalva, továbbá az elosztott hálózatok leírására szolgáló matematikai modellező nyelv, a Petri háló.
A gráfelméleti fogalmak előtt részletesen szó esik az automatákról, és azok kapcsolatáról a strukturális folyamatokra nézve. 

A modellezett folyamatok véges állapotú automatáknak tekinthetők. A dolgozat részletezi ezek megvalósításának gyakori módjait, az azokból adódó előnyöket és hátrányokat. A vizsgálatok kitérnek arra, hogy hogyan lehet kezelni a folyamatokban bekövetkező változásokat, automatikusan jelezni a hibákat, javaslatokat tenni a folyamatok hatékonyabbá tételére, ilyen módon optimalizálva azokat.

Az alkalmazás szerver oldali része C\# programozási nyelven .Net környezetben készül. Az adatok perzisztens tárolásához Microsoft SQL szerver kerül felhasználásra, melyhez a hozzáférést az Entity Framework segíti. A kliens megvalósításához a React nevű JavaScript keretrendszert használja az alkalmazás.

% TODO: Ki lehet majd térni tételesen az egyes fejezetekben szereplő dolgokra.
